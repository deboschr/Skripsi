% \subsection{Java HTTP Client API}
% \label{subsec:0304-httpclient}

Selain Jsoup yang digunakan untuk mengambil dan memproses halaman situs web, sistem ini juga membutuhkan mekanisme pemeriksaan terhadap tautan lain, baik tautan eksternal maupun tautan non-halaman (misalnya berkas gambar, skrip, atau sumber daya lain). Untuk kebutuhan ini dipilih Java HTTP Client API, sebagaimana dijelaskan pada Subbab~\ref{sec:02-java-http-client-api}. API ini merupakan bagian resmi dari Java sejak versi 11, sehingga tidak memerlukan pustaka tambahan dan terintegrasi langsung dengan ekosistem bahasa Java.  

Pemanfaatan Java HTTP Client API sejalan dengan beberapa kebutuhan non-fungsional, antara lain penerapan batas waktu (\textit{timeout}) agar pemeriksaan tidak menggantung terlalu lama, pengendalian laju permintaan melalui konfigurasi \texttt{HttpClient}, serta dukungan eksekusi paralel yang memanfaatkan sifat \textit{thread-safe} dari objek \texttt{HttpClient}. Dari sisi fungsional, API ini memungkinkan sistem untuk memperoleh status kode HTTP dari setiap tautan, yang kemudian ditampilkan dengan label yang jelas pada antarmuka (lihat Subsubbab~\ref{subsec:0303-kebutuhan-fungsional} dan Subsubbab~\ref{subsec:0303-kebutuhan-non-fungsional}).

Beberapa kelas dan method utama dari API ini yang digunakan dalam implementasi adalah sebagai berikut:

\begin{itemize}
  \item \texttt{HttpClient}\\
  Kelas ini dipakai sebagai titik masuk untuk membangun klien HTTP. Objek dibuat melalui \texttt{HttpClient.newBuilder()}, dengan konfigurasi penting seperti \texttt{version()} untuk memilih protokol HTTP/1.1 atau HTTP/2, \texttt{connectTimeout(Duration)} untuk menetapkan batas waktu koneksi, dan \texttt{followRedirects()} untuk menentukan kebijakan pengalihan. Objek yang dihasilkan bersifat \textit{immutable} dan \textit{thread-safe}, sehingga dapat digunakan secara bersamaan oleh banyak \textit{thread} pemeriksaan.

  \item \texttt{HttpRequest}\\
  Kelas ini digunakan untuk menyusun permintaan HTTP yang akan dikirim. Metode \texttt{uri(URI)} menentukan alamat tujuan, sementara \texttt{header()} dipakai untuk menyertakan informasi tambahan seperti \texttt{User-Agent}. Jenis operasi HTTP ditentukan dengan \texttt{method()} atau \texttt{GET()}, sedangkan \texttt{timeout(Duration)} memastikan setiap permintaan memiliki batas waktu pemrosesan.

  \item \texttt{HttpResponse}\\
  Objek ini merepresentasikan hasil tanggapan dari server. Informasi utama yang digunakan adalah kode status HTTP melalui \texttt{statusCode()}, \textit{header} melalui \texttt{headers()}, serta isi respon melalui \texttt{body()}. Data ini akan dicatat dan ditampilkan pada antarmuka, termasuk untuk mengidentifikasi tautan rusak (misalnya dengan status 404) atau tautan yang terblokir.

  \item \texttt{HttpHeaders}\\
  Digunakan untuk membaca seluruh \textit{header} dari respon, misalnya \texttt{Content-Type} atau \texttt{Set-Cookie}. Informasi ini relevan untuk analisis tambahan, namun yang paling penting adalah mendukung pelabelan hasil pemeriksaan agar pengguna dapat memahami konteks status setiap tautan.
\end{itemize}