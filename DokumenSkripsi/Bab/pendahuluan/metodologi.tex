Metodologi penelitian yang digunakan pada tugas akhir ini adalah sebagai berikut:

\begin{enumerate}

    \item Melakukan studi literatur terhadap protokol HTTP, URI, situs web dan \textit{web crawling}.

    \item Melakukan studi literatur dan eksperimen terhadap Jsoup, Java \texttt{HttpClient}, JavaFX dan Gradle.

    \item Menganalisis permasalahan pemeriksaan tautan rusak pada situs web, meninjau perangkat lunak serupa, merumuskan kebutuhan fungsional dan non-fungsional, serta menganalisis teknologi yang akan digunakan dalam pengembangan aplikasi.

    \item Merancang kelas-kelas utama dan antarmuka pengguna perangkat lunak.
    
    \item Mengimplementasikan perangkat lunak pemeriksa tautan rusak situs web.

    \item Melakukan pengujian terhadap beberapa situs web dan membandingkan hasilnya dengan perangkat lunak sejenis untuk mengevaluasi hasil dan kemampuannya dalam pemeriksaan tautan rusak.
    
    \item Menulis dokumen tugas akhir.
    
\end{enumerate}