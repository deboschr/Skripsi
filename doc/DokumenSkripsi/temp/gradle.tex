
Gradle merupakan sistem otomasi pembangunan perangkat lunak (\textit{build automation}) yang digunakan untuk mengelola proses kompilasi, pengujian, dan distribusi perangkat lunak. Gradle dikembangkan sebagai generasi lanjutan dari sistem build sebelumnya seperti Apache Ant dan Maven, dengan tujuan menyediakan mekanisme yang lebih fleksibel dan dapat diprogram.  

Gradle menggunakan \textit{domain-specific language} (DSL) berbasis Groovy atau Kotlin untuk mendefinisikan proses build. Pendekatan ini menggabungkan gaya deklaratif, di mana pengembang menyatakan konfigurasi yang diinginkan, dengan gaya imperatif, di mana logika tambahan dapat ditulis sebagai kode program. Dengan demikian, Gradle memungkinkan penulisan skrip build yang ringkas namun tetap mendukung kustomisasi tingkat lanjut.  

Arsitektur Gradle didesain modular dan berbasis plugin. Plugin menyediakan konfigurasi dan sekumpulan \textit{task} standar sesuai kebutuhan proyek, misalnya untuk membangun aplikasi Java atau mengelola arsip distribusi. Sistem ini juga terintegrasi dengan repositori perangkat lunak seperti Maven Central, sehingga dependensi eksternal dapat diunduh dan dikelola secara otomatis.

\subsection{Konsep Build dan Task}
Dalam pengembangan perangkat lunak, proses \textit{build} mencakup serangkaian langkah seperti kompilasi kode sumber, manajemen dependensi, pengujian otomatis, serta pembuatan artefak distribusi seperti JAR atau WAR. Gradle memodelkan setiap langkah tersebut sebagai \textit{task}, yaitu unit kerja terpisah yang dapat dijalankan secara independen maupun sebagai bagian dari rangkaian tugas lain.  

Setiap task memiliki tindakan utama (\textit{action}) dan dapat didefinisikan dengan ketergantungan terhadap task lain. Dengan cara ini, Gradle membangun sebuah grafik ketergantungan yang menentukan urutan eksekusi seluruh proses build.  

Kode~\ref{lst:gradle-simple-task} memperlihatkan contoh definisi task sederhana dalam Gradle.

\begin{lstlisting}[language=groovy, caption=Contoh task sederhana dalam Gradle, label=lst:gradle-simple-task]
task cetakPesan {
    doLast {
        println 'Ini adalah task pertama saya di Gradle!'
    }
}
\end{lstlisting}

Kode pada Kode~\ref{lst:gradle-simple-task} mendefinisikan sebuah task bernama \texttt{cetakPesan}. Task ini memiliki satu aksi yang dijalankan pada fase \texttt{doLast}, yaitu mencetak pesan ke konsol. Task dapat dijalankan secara langsung dengan perintah \texttt{gradle cetakPesan}, dan Gradle akan mengeksekusi instruksi yang terdefinisi di dalamnya. Dengan pendekatan ini, pengembang dapat mendefinisikan berbagai tahapan build secara modular dan dapat digunakan kembali.


\subsection{Struktur Proyek}
Gradle mendukung struktur proyek yang fleksibel, tetapi umumnya mengikuti konvensi tertentu agar proses build lebih sederhana. Beberapa berkas dan direktori yang umum ditemukan dalam proyek Gradle antara lain:  

\begin{itemize}
    \item \texttt{build.gradle}, berisi deklarasi task, konfigurasi plugin, dan daftar dependensi.
    \item \texttt{settings.gradle}, mendefinisikan nama proyek dan konfigurasi multiproject bila proyek terdiri dari lebih dari satu modul.
    \item \texttt{src/main/java}, direktori utama untuk kode sumber aplikasi Java.
    \item \texttt{src/test/java}, direktori untuk kode pengujian unit.
\end{itemize}

Kode~\ref{lst:gradle-java} memperlihatkan contoh berkas \texttt{build.gradle} untuk proyek Java sederhana.

\begin{lstlisting}[language=groovy, caption=Contoh berkas build.gradle untuk proyek Java, label=lst:gradle-java]
plugins {
    id 'java'
    id 'application'
}

repositories {
    mavenCentral()
}

dependencies {
    implementation 'org.jsoup:jsoup:1.15.4'
    testImplementation 'junit:junit:4.13.2'
}

application {
    mainModule = 'com.unpar.brokenlinkchecker'
    mainClass = 'com.unpar.brokenlinkchecker.Application'
}
\end{lstlisting}

Kode pada Kode~\ref{lst:gradle-java} menunjukkan bagaimana proyek Java didefinisikan dalam Gradle. Pada bagian \texttt{plugins}, diaktifkan plugin \texttt{java} untuk mendukung kompilasi kode Java dan plugin \texttt{application} untuk menjalankan aplikasi. Bagian \texttt{repositories} menentukan bahwa dependensi diambil dari \texttt{mavenCentral}. Selanjutnya, blok \texttt{dependencies} mendefinisikan pustaka eksternal yang digunakan, yaitu \texttt{jsoup} untuk pemrosesan HTML dan \texttt{junit} untuk pengujian. Pada blok \texttt{application}, ditentukan modul utama dan kelas utama yang menjadi titik masuk aplikasi.


\subsection{Struktur Proyek}
Gradle mendukung struktur proyek yang fleksibel, tetapi umumnya mengikuti konvensi tertentu agar proses build lebih sederhana. Beberapa berkas dan direktori yang umum ditemukan dalam proyek Gradle antara lain:  

\begin{itemize}
    \item \texttt{build.gradle}, berisi deklarasi task, konfigurasi plugin, dan daftar dependensi.
    \item \texttt{settings.gradle}, mendefinisikan nama proyek dan konfigurasi multiproject bila proyek terdiri dari lebih dari satu modul.
    \item \texttt{src/main/java}, direktori utama untuk kode sumber aplikasi Java.
    \item \texttt{src/test/java}, direktori untuk kode pengujian unit.
\end{itemize}

Kode~\ref{lst:gradle-java} memperlihatkan contoh berkas \texttt{build.gradle} untuk proyek Java sederhana.

\begin{lstlisting}[language=groovy, caption=Contoh berkas build.gradle untuk proyek Java, label=lst:gradle-java]
plugins {
    id 'java'
    id 'application'
}

repositories {
    mavenCentral()
}

dependencies {
    implementation 'org.jsoup:jsoup:1.15.4'
    testImplementation 'junit:junit:4.13.2'
}

application {
    mainModule = 'com.unpar.brokenlinkchecker'
    mainClass = 'com.unpar.brokenlinkchecker.Application'
}
\end{lstlisting}

Kode pada Kode~\ref{lst:gradle-java} menunjukkan bagaimana proyek Java didefinisikan dalam Gradle. Pada bagian \texttt{plugins}, diaktifkan plugin \texttt{java} untuk mendukung kompilasi kode Java dan plugin \texttt{application} untuk menjalankan aplikasi. Bagian \texttt{repositories} menentukan bahwa dependensi diambil dari \texttt{mavenCentral}. Selanjutnya, blok \texttt{dependencies} mendefinisikan pustaka eksternal yang digunakan, yaitu \texttt{jsoup} untuk pemrosesan HTML dan \texttt{junit} untuk pengujian. Pada blok \texttt{application}, ditentukan modul utama dan kelas utama yang menjadi titik masuk aplikasi.


\subsection{Plugin, Repositori, dan DSL}
Gradle memiliki arsitektur berbasis plugin. Plugin menyediakan konfigurasi dan sekumpulan task yang sesuai dengan jenis proyek tertentu. Sebagai contoh, plugin \texttt{java} menambahkan task untuk kompilasi dan pengujian kode Java, sedangkan plugin \texttt{application} menambahkan task untuk menjalankan aplikasi.  

Dependensi eksternal dikelola melalui sistem repositori. Gradle mendukung berbagai repositori seperti \texttt{mavenCentral}, \texttt{jcenter}, maupun repositori lokal. Dengan pendekatan ini, pustaka pihak ketiga dapat diunduh secara otomatis dan disertakan dalam proses build.  

Gradle menggunakan \textit{domain-specific language} (DSL) berbasis Groovy atau Kotlin untuk menuliskan konfigurasi build. DSL ini memungkinkan pengembang mendefinisikan task, plugin, repositori, dan dependensi dalam bentuk blok kode yang ringkas. Konsep seperti \textit{closure} dan \textit{convention-over-configuration} digunakan untuk menyederhanakan sintaks sekaligus tetap memberikan fleksibilitas dalam kustomisasi alur build.


\subsection{Siklus Eksekusi Gradle}
Proses eksekusi build pada Gradle berlangsung dalam dua fase utama, yaitu fase konfigurasi dan fase eksekusi.  

Pada fase konfigurasi, Gradle membaca seluruh berkas skrip build seperti \texttt{build.gradle} dan \texttt{settings.gradle}. Seluruh definisi task, plugin, dan konfigurasi dievaluasi untuk membangun model tugas beserta ketergantungan antar tugas. Pada tahap ini belum ada task yang dijalankan, melainkan hanya persiapan struktur eksekusi.  

Setelah konfigurasi selesai, Gradle memasuki fase eksekusi. Pada tahap ini, Gradle menjalankan task yang diminta beserta seluruh dependensinya sesuai dengan urutan yang telah ditentukan. Fase eksekusi memastikan setiap task dijalankan satu kali dan hanya jika dibutuhkan, misalnya saat terjadi perubahan pada kode sumber.  

Pemisahan yang jelas antara fase konfigurasi dan fase eksekusi memungkinkan Gradle mengelola alur build secara efisien dan fleksibel, serta memberikan kendali penuh kepada pengembang dalam menuliskan logika build.
