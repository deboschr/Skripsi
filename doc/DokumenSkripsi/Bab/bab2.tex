\chapter{Landasan Teori}
\label{chap:landasan-teori}
Bab ini membahas teori dan teknologi yang mendasari pengembangan perangkat lunak pemeriksa tautan rusak, mencakup protokol HTTP, URI, konsep dasar situs web, \textit{web crawling}, serta pustaka dan teknologi yang digunakan seperti Jsoup, Java HTTP Client API, JavaFX dan Gradle.


\section{HTTP~\cite{RFC9110}}
\label{sec:02-http}
\textit{Hypertext Transfer Protocol} (HTTP) adalah protokol tingkat aplikasi yang menjadi dasar komunikasi pada arsitektur \textit{World Wide Web}. HTTP bekerja dengan pola komunikasi berbasis pesan, di mana sebuah \textit{client} mengirimkan \textit{request} dan \textit{server} menanggapinya dengan \textit{response}. Protokol ini bersifat \textit{stateless}, artinya setiap \textit{request} dapat dipahami secara terpisah tanpa ketergantungan pada interaksi sebelumnya, sehingga \textit{server} tidak diwajibkan menyimpan konteks antar \textit{request}. Secara umum, HTTP dijalankan di atas protokol TCP dengan port standar 80. Untuk komunikasi yang membutuhkan perlindungan, digunakan skema \texttt{https}, yaitu HTTP yang berjalan di atas \textit{Transport Layer Security} (TLS) pada port standar 443. Melalui HTTPS, komunikasi memperoleh jaminan kerahasiaan, autentikasi, dan integritas data.

Tujuan dari sebuah \textit{request} adalah \textit{resource}, yang umumnya diidentifikasi melalui URI~\ref{sec:02-uri}. HTTP tidak membatasi apa yang dimaksud dengan \textit{resource}, melainkan hanya menyediakan antarmuka untuk berinteraksi dengannya. Informasi mengenai \textit{resource} tersebut disampaikan dalam bentuk \textit{representation}, yaitu data beserta metadata yang mencerminkan keadaan \textit{resource} pada waktu tertentu dan dapat ditransmisikan melalui protokol.

Dalam model komunikasi HTTP, peran \textit{client} dan \textit{server} menjadi kunci. \textit{Client} adalah program yang membuka koneksi untuk mengirimkan \textit{request}, sedangkan \textit{server} menerima koneksi dan memberikan \textit{response}. Bentuk \textit{client} yang paling umum adalah \textit{user agent}, seperti \textit{web browser}, \textit{spider} (robot \textit{web-traversing}), hingga \textit{command-line tools}. Di sisi lain, \textit{origin server} adalah program yang dapat menghasilkan \textit{response} otoritatif untuk sebuah \textit{resource}.

Untuk meningkatkan kinerja, HTTP juga memanfaatkan \textit{caches}. Sebuah \textit{cache} menyimpan \textit{response} sebelumnya yang bersifat \textit{cacheable}, sehingga \textit{request} yang sama di kemudian hari dapat dijawab lebih cepat tanpa perlu menghubungi kembali \textit{origin server}. Mekanisme ini membantu mengurangi waktu \textit{response} sekaligus menekan konsumsi \textit{bandwidth}. 




% \subsection{Struktur Pesan HTTP}
% \label{subsec:0201-struktur-pesan-http}


\subsection{Metode HTTP}
\label{subsec:0201-metode-http}

Metode HTTP berfungsi untuk menunjukkan tujuan dari \textit{request} yang dibuat oleh \textit{client} dan hasil sukses apa yang diharapkan dari \textit{request} tersebut. Metode HTTP memiliki beberapa sifat umum, diantaranya adalah \textit{safe} dan \textit{idempotent}. Sebuah metode dikatakan \textit{safe} apabila semantik yang didefinisikan bersifat \textit{read-only}, yaitu \textit{client} tidak meminta dan tidak mengharapkan adanya perubahan keadaan pada \textit{origin server} akibat penerapan metode tersebut. Metode yang termasuk \textit{safe} adalah \texttt{GET}, \texttt{HEAD}, \texttt{OPTIONS}, dan \texttt{TRACE}. Sebuah metode disebut \textit{idempotent} apabila dampak yang dimaksudkan pada \textit{server} dari beberapa \textit{request} identik, sama dengan dampak dari satu \textit{request}. Metode \texttt{PUT}, \texttt{DELETE}, serta seluruh metode aman adalah \textit{idempotent}. Berikut ini adalah daftar beberapa metode HTTP:

\begin{itemize}
    \item \textbf{GET}: Metode ini digunakan untuk meminta transfer representasi terkini dari sumber daya target.
  
    \item \textbf{HEAD}: Metode ini identik dengan \texttt{GET}, tetapi \textit{server} tidak boleh mengirimkan konten dalam \textit{response}. \texttt{HEAD} digunakan untuk memperoleh metadata dari representasi yang dipilih tanpa harus mentransfer data representasi itu sendiri.
  
    \item \textbf{POST}: Metode ini digunakan untuk meminta sumber daya target memproses representasi yang disertakan dalam \textit{request} sesuai dengan semantik khusus yang dimiliki oleh sumber daya tersebut.
  
    \item \textbf{PUT}: Metode ini digunakan untuk meminta agar keadaan dari sumber daya target dibuat atau diganti dengan keadaan yang ditentukan oleh representasi yang disertakan dalam isi pesan \textit{request}.
  
    \item \textbf{DELETE}: Metode ini digunakan untuk meminta agar \textit{origin server} menghapus asosiasi antara sumber daya target dengan fungsionalitasnya saat ini.
  
    \item \textbf{OPTIONS}: Metode ini digunakan untuk meminta informasi mengenai opsi komunikasi yang tersedia bagi sumber daya target, baik pada \textit{origin server} maupun perantara. Metode ini memungkinkan \textit{client} mengetahui opsi dan/atau persyaratan yang terkait dengan sebuah sumber daya, atau kemampuan dari sebuah \textit{server}, tanpa menyiratkan adanya tindakan terhadap sumber daya tersebut.
  
\end{itemize}



\subsection{Kode Status HTTP}
\label{subsec:0201-kode-status-http}

Kode status HTTP adalah bagian dari baris awal pada \textit{response} \textit{server} yang menunjukkan hasil pemrosesan terhadap suatu \textit{request}. Kode ini terdiri dari tiga digit numerik dan dikelompokkan ke dalam lima kelas utama berdasarkan digit pertamanya: informasi (1xx), keberhasilan (2xx), pengalihan (3xx), kesalahan dari \textit{client} (4xx), dan kesalahan dari \textit{server} (5xx).

\subsubsection{\textit{Informational} 1xx}
\label{subsubsec:020104-infotmational-1xx}

Kode-kode pada kelas ini menunjukkan bahwa \textit{request} telah diterima dan sedang diproses, tetapi belum ada \textit{response} final.

\begin{itemize}
    \item \textbf{100 (\textit{Continue})}: Menunjukkan bahwa bagian awal dari \textit{request} telah diterima dan belum ditolak oleh \textit{server}. \textit{Server} bermaksud untuk mengirimkan \textit{response} akhir setelah seluruh \textit{request} diterima dan diproses.
  
    \item \textbf{101 (\textit{Switching Protocols})}: Menunjukkan bahwa \textit{server} memahami dan bersedia memenuhi \textit{request} \textit{client} untuk beralih protokol aplikasi pada koneksi yang sama.
  
    \item \textbf{102 (\textit{Processing})}: Menunjukan bahwa \textit{server} telah menerima \textit{request} sepenuhnya, namun pemrosesan terhadap \textit{request} tersebut belum selesai.~\cite{RFC2518}
  
    \item \textbf{103 (\textit{Early Hints})}: Digunakan oleh \textit{server} untuk memberikan \textit{response} awal melalui \textit{header} kepada \textit{client} sebelum \textit{response} akhir tersedia.~\cite{RFC8297}
  
\end{itemize}

\subsubsection{\textit{Successful} 2xx}
\label{subsubsec:201004-successful-2xx}

Kode-kode pada kelas ini menunjukkan bahwa \textit{request} telah diterima, dipahami, dan diproses dengan sukses.

\begin{itemize}
    \item \textbf{200 (\textit{OK})}: Menunjukkan bahwa \textit{request} berhasil diproses.
  
    \item \textbf{201 (\textit{Created})}: Menunjukkan bahwa \textit{request} berhasil dan menghasilkan satu atau lebih sumber daya baru.
  
    \item \textbf{202 (\textit{Accepted})}: Menunjukkan bahwa \textit{request} telah diterima untuk diproses, tetapi pemrosesan belum selesai.
  
    \item \textbf{203 (\textit{Non-Authoritative Information})}: Menunjukkan bahwa \textit{request} berhasil, tetapi konten yang dikirim telah dimodifikasi dari \textit{response} 200 (\textit{OK}) oleh \textit{transforming proxy}. \textit{Transforming proxy} adalah \textit{proxy} yang melakukan modifikasi terhadap representasi pesan, baik dengan cara mengubah format, menambahkan informasi, maupun menghapus sebagian konten, sehingga representasi yang dikirimkan kepada \textit{client} tidak identik dengan yang dikirimkan oleh \textit{origin server}.
  
    \item \textbf{204 (\textit{No Content})}: Menunjukkan bahwa \textit{request} berhasil, tetapi tidak ada konten tambahan untuk dikirim dalam \textit{response}.
  
    \item \textbf{205 (\textit{Reset Content})}: Menunjukkan bahwa \textit{server} telah berhasil memenuhi \textit{request} dan menginginkan agar \textit{user agent} mereset tampilan dokumen yang menyebabkan \textit{request} dikirim, ke keadaan awal sebagaimana diterima dari \textit{origin server}.
  
    \item \textbf{206 (\textit{Partial Content})}: Menunjukkan bahwa \textit{server} berhasil memenuhi \textit{range request} dengan mengirimkan sebagian dari representasi sumber daya. Klien harus memeriksa \texttt{Content-Type} dan \texttt{Content-Range} untuk mengetahui bagian mana yang dikirim dan apakah diperlukan \textit{request} tambahan. \textit{Range request} didefinisikan sebagai \textit{request} HTTP yang menggunakan \textit{header} \texttt{Range} untuk meminta sebagian dari representasi data.
  
    \item \textbf{207 (\textit{Multi-Status})}: Digunakan untuk memberikan status terhadap beberapa operasi independen dalam satu \textit{request}.~\cite{RFC4918}
  
  
    \item \textbf{226 (IM \textit{Used})}: Menunjukkan bahwa \textit{server} telah berhasil memenuhi \textit{request} \texttt{GET} terhadap suatu sumber daya, dan representasi yang dikembalikan merupakan hasil dari satu atau lebih \textit{instance manipulations} yang diterapkan pada \textit{instance} saat ini. \textit{Instance manipulation} adalah operasi terhadap satu atau lebih \textit{instance} yang dapat menghasilkan representasi \textit{instance} dikirimkan dari \textit{server} ke klien dalam bentuk bagian-bagian terpisah atau melalui lebih dari satu pesan \textit{response}.~\cite{RFC3229}
  
\end{itemize}

\subsubsection{\textit{Redirection} 3xx}
\label{subsubsec:020104-redirection-3xx}

Kode-kode pada kelas ini menunjukkan bahwa \textit{client} harus melakukan langkah tambahan untuk menyelesaikan \textit{request}, seperti mengikuti \textit{redirect}.

\begin{itemize}

    \item \textbf{300 (\textit{Multiple Choices})}: Menunjukkan bahwa sumber daya target memiliki lebih dari satu representasi, masing-masing dengan \textit{identifier} yang lebih spesifik. Informasi mengenai alternatif tersebut diberikan agar pengguna atau \textit{user agent} dapat memilih representasi yang diinginkan dengan mengarahkan permintaannya ke salah satu dari \textit{identifier} tersebut.
  
    \item \textbf{301 (\textit{Moved Permanently})}: Menunjukkan bahwa sumber daya target telah diberikan URI baru yang bersifat permanen, dan semua referensi di masa depan sebaiknya menggunakan URI baru tersebut.
  
    \item \textbf{302 (\textit{Found})}: Menunjukkan bahwa sumber daya target sementara berada di bawah URI yang berbeda. Karena lokasi tersebut dapat berubah, \textit{client} tetap sebaiknya menggunakan URI asli untuk \textit{request} di masa depan.
  
    \item \textbf{303 (\textit{See Other})}: Menunjukkan bahwa \textit{server} mengarahkan \textit{user agent} ke sumber daya lain, sebagaimana ditentukan dalam \textit{header} \texttt{Location}, yang dimaksudkan untuk memberikan \textit{response} tidak langsung terhadap \textit{request} asli.
  
    \item \textbf{304 (\textit{Not Modified})}: Menunjukan bahwa \textit{server} telah menerima \textit{conditional request} untuk metode \texttt{GET} atau \texttt{HEAD} yang seharusnya menghasilkan \textit{response} 200 (\textit{OK}), tetapi kondisi yang diberikan bernilai salah. Artinya, \textit{client} sudah memiliki representasi yang valid, sehingga \textit{server} tidak perlu mengirim ulang. Klien dapat menggunakan salinan yang sudah dimiliki seolah-olah \textit{server} memberikan \textit{response} 200 (\textit{OK}).
 
    \item \textbf{307 (\textit{Temporary Redirect})}: Menunjukkan bahwa sumber daya target sementara berada di bawah URI yang berbeda.
  
    \item \textbf{308 (\textit{Permanent Redirect})}: Menunjukkan bahwa sumber daya target telah dipindahkan secara permanen ke URI baru, dan semua referensi berikutnya sebaiknya menggunakan URI baru tersebut.
  
\end{itemize}

\subsubsection{\textit{Client Error} 4xx}
\label{subsubsec:020104-client-error-4xx}

Kode-kode pada kelas ini menunjukkan bahwa telah terjadi kesalahan di sisi \textit{client}. Berikut adalah daftar kode pada kelas ini:

\begin{itemize}
    \item \textbf{400 (\textit{Bad Request})}: Menunjukkan bahwa \textit{server} tidak dapat atau tidak mau memproses \textit{request} karena dianggap sebagai kesalahan dari \textit{client}, seperti sintaks \textit{request} salah, format pesan tidak valid, atau rute \textit{request} yang menyesatkan.
  
    \item \textbf{401 (\textit{Unauthorized})}: Menunjukkan bahwa \textit{request} tidak dapat dijalankan karena tidak memiliki kredensial autentikasi yang sah untuk sumber daya yang diminta.
  
    \item \textbf{402 (\textit{Payment Required})}: Kode ini disediakan untuk penggunaan di masa depan.
  
    \item \textbf{403 (\textit{Forbidden})}: Menunjukkan bahwa \textit{server} memahami \textit{request} tetapi menolak untuk memenuhinya.
  
    \item \textbf{404 (\textit{Not Found})}: Menunjukkan bahwa sumber daya yang diminta tidak ditemukan.
  
    \item \textbf{405 (\textit{Method Not Allowed})}: Menunjukkan bahwa metode HTTP pada \textit{request} dikenali oleh \textit{server}, tetapi tidak diizinkan untuk digunakan pada sumber daya tersebut.
  
    \item \textbf{406 (\textit{Not Acceptable})}: Menunjukkan bahwa sumber daya target tidak memiliki representasi yang sesuai dengan preferensi \textit{client} berdasarkan \textit{header} negosiasi konten.
  
    \item \textbf{407 (\textit{Proxy Authentication Required})}: Mirip dengan kode status 401, tetapi digunakan ketika \textit{client} harus melakukan autentikasi terlebih dahulu kepada \textit{proxy}.
  
    \item \textbf{408 (\textit{Request Timeout})}: Menunjukkan bahwa \textit{server} tidak menerima pesan \textit{request} yang lengkap dalam jangka waktu yang sudah disiapkannya untuk menunggu.
  
    \item \textbf{409 (\textit{Conflict})}: Menunjukkan bahwa \textit{request} tidak dapat diselesaikan karena terjadi konflik dengan keadaan sumber daya target saat ini.
  
    \item \textbf{410 (\textit{Gone})}: Menunjukkan bahwa akses ke sumber daya target sudah tidak tersedia lagi di \textit{origin server}, dan kondisi ini kemungkinan bersifat permanen.
  
    \item \textbf{411 (\textit{Length Required})}: Menunjukkan bahwa \textit{server} menolak menerima \textit{request} yang tidak memiliki \textit{header} \texttt{Content-Length}.
  
    \item \textbf{412 (\textit{Precondition Failed})}: Menunjukkan bahwa satu atau lebih kondisi yang dikirim dalam \textit{header request} bernilai salah ketika diperiksa oleh \textit{server}.
  
    \item \textbf{413 (\textit{Content Too Large})}: Menunjukkan bahwa \textit{server} menolak memproses \textit{request} karena ukuran konten lebih besar daripada yang bersedia atau mampu diproses \textit{server}.
  
    \item \textbf{414 (URI \textit{Too Long})}: Menunjukkan bahwa \textit{server} menolak memproses \textit{request} karena URI target terlalu panjang untuk ditafsirkan.
  
    \item \textbf{415 (\textit{Unsupported Media Type})}: Menunjukkan bahwa \textit{server} menolak memproses \textit{request} karena isi \textit{request} berada dalam format yang tidak didukung oleh metode pada sumber daya target.
  
    \item \textbf{416 (\textit{Range Not Satisfiable})}: 
  
    \item \textbf{417 (\textit{Expectation Failed})}: Menunjukkan bahwa ekspektasi yang ditentukan dalam \textit{header} \texttt{Expect} pada \textit{request}, tidak dapat dipenuhi oleh setidaknya salah satu \textit{server} yang menerima \textit{request}.
  
    \item \textbf{421 (\textit{Misdirected Request})}: Menunjukkan bahwa \textit{request} diarahkan ke \textit{server} yang tidak mampu atau tidak mau memberikan \textit{response} otoritatif untuk URI target.
  
    \item \textbf{422 (\textit{Unprocessable Content})}: Menunjukkan bahwa \textit{server} memahami jenis konten \textit{request} dan sintaks \textit{request} benar, tetapi \textit{server} tidak dapat memproses instruksi di dalamnya.
  
    \item \textbf{423 (\textit{Locked})}: Menunjukkan bahwa sumber atau tujuan dari suatu metode sedang terkunci.~\cite{RFC4918}
  
    \item \textbf{424 (\textit{Failed Dependency})}: Menunjukkan bahwa metode tidak dapat dijalankan pada sumber daya karena tindakan yang menjadi prasyarat telah gagal.~\cite{RFC4918}
  
    \item \textbf{425 (\textit{Too Early})}: Menunjukkan bahwa \textit{server} menolak memproses \textit{request} karena khawatir \textit{request} tersebut dapat diulang (\textit{replayed}).~\cite{RFC8470}
  
    \item \textbf{426 (\textit{Upgrade Required})}: Menunjukkan bahwa \textit{server} menolak menjalankan \textit{request} menggunakan protokol saat ini, tetapi bersedia melakukannya setelah \textit{client} beralih ke protokol lain.
  
    \item \textbf{428 (\textit{Precondition Required})}: Menunjukkan bahwa \textit{server} mewajibkan \textit{request} disertai kondisi tertentu.~\cite{RFC6585}
  
    \item \textbf{429 (\textit{Too Many Requests})}: Menunjukkan bahwa \textit{client} telah mengirim terlalu banyak \textit{request} dalam jangka waktu tertentu.~\cite{RFC6585}
  
    \item \textbf{431 (\textit{Request Header Fields Too Large})}: Menunjukkan bahwa \textit{server} menolak \textit{request} karena ukuran \textit{header} terlalu besar.~\cite{RFC6585}
  
    \item \textbf{451 (\textit{Unavailable For Legal Reasons})}: Sumber daya dibatasi karena alasan hukum.~\cite{RFC7725}
  
\end{itemize}

\subsubsection{\textit{Server Error} 5xx}
\label{subsubsec:020104-server-error-5xx}

Kode-kode pada kelas ini menunjukkan bahwa \textit{server} menyadari bahwa terjadi kesalahan di sisi \textit{server}. Berikut adalah daftar kode pada kelas ini:

\begin{itemize}

    \item \textbf{500 (\textit{Internal Server Error})}: Menunjukkan bahwa \textit{server} mengalami kondisi tak terduga yang menghalangi \textit{server} untuk dapat memenuhi.
  
    \item \textbf{501 (\textit{Not Implemented})}: Menunjukkan bahwa \textit{server} tidak mendukung fungsionalitas yang diperlukan untuk memenuhi \textit{request} dari \textit{client}. Kode ini digunakan ketika \textit{server} tidak mengenali metode HTTP yang digunakan, dan tidak memiliki kemampuan untuk mendukung metode tersebut pada sumber daya mana pun.
  
    \item \textbf{502 (\textit{Bad Gateway})}: Menunjukkan bahwa \textit{server} ketika bertindak sebagai \textit{gateway} atau \textit{proxy}, menerima \textit{response} tidak yang valid dari \textit{server} lain yang diaksesnya saat mencoba memenuhi \textit{request}.
  
    \item \textbf{503 (\textit{Service Unavailable})}: Menunjukkan bahwa \textit{server} untuk sementara tidak dapat menangani \textit{request}. Kode ini digunakan ketika \textit{server} sedang kelebihan beban atau sedang dilakukan pemeliharaan. \textit{Server} dapat menyertakan \textit{header} \texttt{Retry-After} untuk memberi tahu \textit{client} berapa lama sebaiknya menunggu sebelum mencoba lagi \textit{request}.
  
    \item \textbf{504 (\textit{Gateway Timeout})}: Menunjukkan bahwa \textit{server} ketika bertindak sebagai \textit{gateway} atau \textit{proxy}, tidak menerima \textit{response} tepat waktu dari \textit{server} lain yang perlu diakses untuk menyelesaikan \textit{request}.
  
    \item \textbf{505 (\textit{HTTP Version Not Supported})}: Menunjukkan bahwa \textit{server} tidak mendukung atau menolak mendukung versi HTTP yang digunakan dalam \textit{request}.
  
    \item \textbf{506 (\textit{Variant Also Negotiates})}: Menunjukkan bahwa \textit{server} memiliki kesalahan konfigurasi internal. Hal ini terjadi ketika sumber daya yang dipilih sebagai varian ternyata juga dikonfigurasi untuk melakukan \textit{transparent content negotiation}, yaitu mekanisme untuk secara otomatis memilih versi terbaik dari sebuah sumber daya yang memiliki banyak varian di bawah satu URL. Akibatnya, sumber daya tersebut tidak bisa menjadi titik akhir yang valid dalam proses negosiasi konten.~\cite{RFC2295}
  
    \item \textbf{507 (\textit{Insufficient Storage})}: Menunjukkan bahwa metode tidak dapat dijalankan pada sumber daya karena \textit{server} tidak mampu menyimpan representasi yang dibutuhkan untuk menyelesaikan \textit{request} dengan sukses.~\cite{RFC4918}
  
    \item \textbf{508 (\textit{Loop Detected})}: Menunjukkan bahwa \textit{server} menghentikan suatu operasi karena mendeteksi adanya \textit{infinite loop} ketika memproses \textit{request} dengan \textit{header} \texttt{Depth:infinity}.~\cite{RFC5842}
  
    \item \textbf{511 (\textit{Network Authentication Required})}: Menunjukkan bahwa \textit{client} harus melakukan autentikasi terlebih dahulu untuk mendapatkan akses jaringan.~\cite{RFC6585}
  
\end{itemize}




\section{URI~\cite{RFC3986}}
\label{sec:02-uri}

\textit{Uniform Resource Identifier} (URI) adalah string karakter yang digunakan untuk mengidentifikasi suatu sumber daya. Identifikasi tersebut dapat berupa lokasi, nama, atau kombinasi keduanya. Spesifikasi URI ditetapkan dalam RFC~3986 yang mendefinisikan aturan sintaks, komponen, serta mekanisme representasi sumber daya.

RFC~3986 mendeskripsikan URI sebagai konsep umum yang memiliki beberapa bentuk. Dua bentuk yang paling dikenal adalah \textit{Uniform Resource Locator} (URL) dan \textit{Uniform Resource Name} (URN). URL adalah URI yang menyertakan informasi lokasi dan mekanisme akses terhadap sumber daya, contohnya \texttt{http://unpar.ac.id/index.html}. URN adalah URI yang berfungsi sebagai nama tetap suatu sumber daya tanpa menyertakan lokasi aksesnya, contohnya \texttt{urn:isbn:0451450523}. RFC juga membedakan URI yang bersifat hierarkis dan URI yang bersifat opak. URI hierarkis memiliki komponen seperti \textit{authority} dan \textit{path}, misalnya skema \texttt{http}. URI opak tidak dapat diuraikan ke dalam komponen hierarkis, misalnya skema \texttt{mailto}.

\subsection{Struktur Sintaks URL}
\label{subsec:0202-struktur-url}

URL, sebagai bentuk URI yang bersifat hierarkis, memiliki struktur sintaks yang terdiri atas beberapa komponen. Format umum penulisan URL ditunjukkan sebagai berikut:
\begin{center}
\texttt{scheme:[//[user:password@]host[:port]]path[?query][\#fragment]}
\end{center}

Komponen-komponen dalam URL dijelaskan sebagai berikut:
\begin{itemize}
  \item \textbf{Scheme}\\
  Bagian ini dituliskan pada awal URL dan diakhiri dengan tanda titik dua (\texttt{:}). Scheme menunjukkan protokol atau mekanisme akses yang digunakan, contohnya \texttt{http}, \texttt{https}, \texttt{ftp}, atau \texttt{mailto}.
  
  \item \textbf{Authority}\\
  Bagian ini bersifat opsional dan diawali dengan dua garis miring (\texttt{//}). Authority dapat terdiri atas tiga subkomponen, yaitu \textit{userinfo}, \textit{host}, dan \textit{port}. \textit{Userinfo} berisi informasi pengguna dengan format \texttt{user:password@} yang digunakan dalam beberapa skema seperti FTP. \textit{Host} adalah lokasi sumber daya yang dapat berupa nama domain atau alamat IP. \textit{Port} adalah nomor port koneksi, misalnya \texttt{80} untuk HTTP atau \texttt{443} untuk HTTPS. Jika port tidak dituliskan, maka \textit{port default} dari skema tersebut digunakan.
  
  \item \textbf{Path}\\
  Bagian ini menunjukkan jalur menuju sumber daya pada host. Path terdiri atas segmen-segmen yang dipisahkan tanda garis miring (\texttt{/}). Di dalamnya dapat terdapat \textit{dot-segments}, yaitu segmen khusus berupa \texttt{.} yang merujuk ke direktori saat ini dan \texttt{..} yang merujuk ke direktori induk.
  
  \item \textbf{Query}\\
  Bagian ini bersifat opsional dan diawali dengan tanda tanya (\texttt{?}). Query berisi parameter dalam format pasangan \texttt{key=value}. Jika terdapat lebih dari satu parameter, maka masing-masing dipisahkan dengan tanda \texttt{\&}.
  
  \item \textbf{Fragment}\\
  Bagian ini bersifat opsional dan diawali dengan tanda pagar (\texttt{\#}). Fragment digunakan untuk merujuk ke bagian tertentu dari sumber daya. Informasi ini tidak dikirim ke server, melainkan diproses oleh agen pengguna seperti browser.
\end{itemize}

\subsection{Kategori Karakter dan \textit{Percent-Encoding}}
\label{subsec:0202-karakter-percent-encoding}

RFC~3986 menetapkan kategori karakter yang dapat digunakan dalam URI, beserta aturan pengkodeannya.
\begin{itemize}
  \item \textbf{Unreserved characters}\\
  Karakter ini dapat digunakan langsung tanpa pengkodean tambahan. Termasuk di dalamnya huruf alfabet A sampai Z dan a sampai z, digit angka 0 sampai 9, serta simbol \texttt{-}, \texttt{\_}, \texttt{.}, dan \texttt{\textasciitilde}.
  
  \item \textbf{Reserved characters}\\
  Karakter ini memiliki fungsi khusus tergantung pada konteks penggunaannya. RFC membagi reserved characters menjadi dua kelompok. General delimiters meliputi \texttt{:}, \texttt{/}, \texttt{?}, \texttt{\#}, \texttt{[}, \texttt{]}, dan \texttt{@}. Subcomponent delimiters meliputi \texttt{!}, \texttt{\$}, \texttt{\&}, \texttt{'}, \texttt{(}, \texttt{)}, \texttt{*}, \texttt{+}, \texttt{,}, \texttt{;}, dan \texttt{=}. Jika karakter ini digunakan di luar konteks yang diperbolehkan, maka harus dikodekan.
  
  \item \textbf{Karakter lain}\\
  Karakter non-ASCII, spasi, dan simbol lain yang tidak termasuk dalam kategori di atas tidak dapat digunakan secara langsung. Karakter ini harus dikodekan sebelum dapat dimasukkan ke dalam URI.
\end{itemize}

Pengkodean karakter dilakukan dengan mekanisme \textit{percent-encoding}. Mekanisme ini menggantikan karakter dengan representasi heksadesimalnya dalam format \texttt{\%HH}, di mana \texttt{HH} adalah nilai ASCII karakter tersebut. Sebagai contoh, spasi ditulis sebagai \texttt{\%20}, tanda kutip ganda sebagai \texttt{\%22}, dan tanda pagar sebagai \texttt{\%23}. RFC juga mengatur bahwa karakter unreserved yang tidak perlu dikodekan sebaiknya dituliskan dalam bentuk aslinya. Selain itu, digit heksadesimal dalam \textit{percent-encoding} harus ditulis dengan huruf besar.

\subsection{Referensi Absolut dan Relatif}
\label{subsec:0202-referensi-url}

URL dapat berupa referensi absolut maupun relatif. URL absolut mencantumkan seluruh komponen utama termasuk scheme dan authority, sehingga dapat diinterpretasikan secara mandiri. Contohnya adalah \texttt{https://unpar.ac.id/page.html}. URL relatif tidak mencantumkan scheme atau authority dan hanya menyertakan path, query, atau fragment. Contohnya adalah \texttt{images/logo.png}. Selain itu terdapat pula same-document reference, yaitu URL yang hanya berisi fragment untuk merujuk ke bagian tertentu dalam dokumen yang sama, contohnya \texttt{\#section2}.

RFC~3986 mendefinisikan mekanisme resolusi yang digunakan untuk menyusun URL absolut dari sebuah referensi relatif. Resolusi dilakukan dengan menentukan base URI, menggabungkannya dengan URL relatif, lalu menyederhanakan hasilnya. Salah satu tahap penting dalam resolusi adalah penghapusan dot-segments. Prosedur ini menghilangkan segmen \texttt{.} dan \texttt{..} agar path hasil resolusi berada dalam bentuk yang ringkas dan tidak ambigu.

\subsection{Normalisasi URL}
\label{subsec:0202-normalisasi-url}

Normalisasi adalah proses menyeragamkan URL ke bentuk kanonik sebagaimana diatur dalam RFC~3986. Proses ini diperlukan agar URL yang secara sintaksis berbeda tetapi menunjuk ke sumber daya yang sama dapat diperlakukan setara. Bentuk normalisasi meliputi:
\begin{itemize}
  \item \textbf{Normalisasi huruf}\\
  Bagian scheme dan host dituliskan dalam huruf kecil. Sebagai contoh, \texttt{HTTP://Unpar.ac.id} dinormalisasi menjadi \texttt{http://unpar.ac.id}.
  
  \item \textbf{Normalisasi percent-encoding}\\
  Digit heksadesimal dalam percent-encoding harus ditulis dengan huruf besar. Selain itu, karakter unreserved yang dituliskan dalam bentuk percent-encoding diganti dengan bentuk aslinya. Sebagai contoh, \texttt{\%7E} dinormalisasi menjadi \texttt{\~{}}.
  
  \item \textbf{Normalisasi path}\\
  Segmen khusus berupa \texttt{.} dan \texttt{..} dihapus sesuai prosedur agar path menjadi ringkas. Sebagai contoh, \texttt{/a/b/../c/./d.html} dinormalisasi menjadi \texttt{/a/c/d.html}.
  
  \item \textbf{Penghapusan port default}\\
  Jika nomor port yang dituliskan sesuai dengan \textit{port default} dari skema, maka port tersebut dihapus. Contohnya, \texttt{http://unpar.ac.id:80} dinormalisasi menjadi \texttt{http://unpar.ac.id}.
\end{itemize}


\section{\textit{Web Crawling}~\cite{liu:11:webdatamining}}
\label{sec:02-web-crawling}

Analisis pada bagian ini merujuk pada teori mengenai \textit{web crawling} yang dibahas pada Subbab~\ref{sec:02-web-crawling} serta teori HTML pada Subbab~\ref{sec:02-html}. Tujuan utamanya adalah mengidentifikasi permasalahan yang muncul dalam proses \textit{crawling} halaman web sebagai dasar pemeriksaan tautan rusak. Permasalahan tersebut mencakup strategi \textit{crawling}, jenis \textit{crawler} yang digunakan, tantangan teknis yang dihadapi, serta keterkaitannya dengan struktur HTML sebagai sumber tautan.

\subsubsection*{Strategi \textit{crawling}}
Pada Subbab~\ref{sec:02-web-crawling} dijelaskan bahwa strategi \textit{crawling} dapat dilakukan dengan \textit{breadth-first crawling} maupun \textit{na\"{i}ve best-first crawling}. Keduanya memiliki karakteristik yang berbeda.  

\begin{itemize}
  \item \textbf{\textit{Breadth-first crawling}} menyapu halaman secara merata dari titik awal. Strategi ini cocok digunakan untuk aplikasi pemeriksa tautan, karena setiap halaman pada host yang sama dianggap memiliki tingkat kepentingan yang setara. Dengan cara ini, cakupan situs dapat diperoleh lebih menyeluruh tanpa harus memikirkan bobot prioritas antar tautan.
  \item \textbf{\textit{Na\"{i}ve best-first crawling}} menggunakan \textit{priority queue} berbasis skor untuk menentukan halaman yang akan diambil lebih dahulu. Strategi ini bermanfaat bila ada kriteria relevansi yang jelas, misalnya hanya ingin menekankan halaman dengan potensi informasi lebih tinggi. Namun, dalam konteks aplikasi pemeriksa tautan rusak, pendekatan ini tidak relevan. Semua halaman pada satu host dianggap sama penting, sehingga tidak ada dasar untuk memberikan skor prioritas.
\end{itemize}

Dengan demikian, strategi yang logis untuk aplikasi ini adalah \textit{breadth-first crawling}.

\subsubsection*{Jenis Crawler}
Subbab~\ref{sec:02-web-crawling} juga membahas jenis-jenis \textit{crawler}, seperti \textit{universal crawler}, \textit{focused crawler}, dan \textit{topical crawler}.  

\begin{itemize}
  \item \textbf{\textit{Universal crawler}} mencoba merayapi seluruh bagian web dalam skala luas. Jenis ini tidak sesuai untuk aplikasi pemeriksa tautan karena lingkupnya terlalu besar dan tidak efisien.
  \item \textbf{\textit{Focused crawler}} membatasi diri pada kriteria tertentu. Pendekatan ini relevan karena aplikasi ini hanya perlu merayapi halaman dengan host yang sama. Dengan cara ini, proses \textit{crawling} lebih terarah dan sumber daya tidak terbuang pada tautan eksternal.
  \item \textbf{\textit{Topical crawler}} memfokuskan \textit{crawling} pada topik tertentu. Jenis ini biasanya digunakan untuk pengumpulan konten tematik. Untuk pemeriksa tautan, pendekatan ini tidak diperlukan karena tujuan utamanya adalah validitas tautan, bukan isi konten.
\end{itemize}

Dengan pertimbangan tersebut, aplikasi pemeriksa tautan lebih sesuai menggunakan pendekatan \textit{focused crawling} dengan batasan pada domain yang sama.

\subsubsection*{Tantangan Teknis}
Subbab~\ref{subsec:0204-tantangan-crawling} menjelaskan sejumlah tantangan yang umum dihadapi dalam \textit{crawling}. Dalam sistem yang dikembangkan, setiap tantangan tersebut ditinjau kembali dan ditetapkan keputusan penerapannya sebagai berikut:

\begin{itemize}
  \item \textbf{\textit{Fetching}}: dalam sistem ini akan diterapkan pengaturan \textit{timeout} agar proses tidak berhenti terlalu lama pada tautan yang tidak merespons. Selain itu, akan ada jeda antar permintaan sebagai bentuk pengendalian agar server tidak terbebani dan tidak memblokir pemeriksaan.
  
  \item \textbf{\textit{Parsing}}: dalam sistem ini akan digunakan Jsoup untuk mengurai HTML. Parser ini dipilih karena mampu menangani dokumen dengan struktur yang tidak sempurna, sehingga tautan tetap dapat diekstrak. Namun, tautan yang dihasilkan secara dinamis melalui JavaScript tidak akan diperiksa karena sistem hanya memproses HTML statis.
  
  \item \textbf{\textit{Link Extraction} dan \textit{Canonicalization}}: dalam sistem ini semua URL yang ditemukan akan dinormalisasi. Langkah ini dilakukan agar tidak ada duplikasi, misalnya akibat perbedaan huruf besar, tanda garis miring di akhir, atau adanya fragmen. Dengan begitu, satu sumber daya tidak akan dianggap berbeda hanya karena variasi penulisan.
  
  \item \textbf{\textit{Repository}}: dalam sistem ini semua URL yang sudah ditemukan, baik halaman maupun sumber daya lain seperti gambar, skrip, dan stylesheet, akan disimpan di dalam repository. Dengan cara ini, tidak ada tautan yang sama diperiksa ulang dan proses pemeriksaan menjadi lebih efisien.
  
  \item \textbf{\textit{Spider Trap} dan \textit{Infinite Loops}}: dalam sistem ini tidak akan diterapkan mekanisme khusus untuk mendeteksi pola tautan tak terbatas. Namun, repository sudah cukup untuk mencegah kunjungan berulang pada URL yang sama. Pengecualian hanya berlaku untuk kasus \textit{redirect loop}, yang tetap akan diperiksa agar sistem tidak terjebak mengikuti \textit{redirect} tanpa akhir.
  
  \item \textbf{\textit{Concurrency}}: dalam sistem ini \textit{concurrency} akan diterapkan, tetapi hanya pada tahap pemeriksaan tautan (HEAD atau GET) agar proses lebih cepat. Untuk proses \textit{crawling} halaman dan parsing HTML tetap dilakukan secara berurutan agar hasil ekstraksi lebih terkontrol. Jumlah permintaan paralel akan dibatasi supaya tidak menimbulkan pemblokiran atau respons “too many requests” dari server.
\end{itemize}



\subsubsection*{Etika \textit{crawling}}
Subbab~\ref{subsec:0204-etika-crawling} menjelaskan bahwa aktivitas \textit{crawling} tidak hanya berkaitan dengan tantangan teknis, tetapi juga harus memperhatikan etika agar tidak menimbulkan masalah bagi pemilik situs. Salah satu pedoman yang tersedia adalah file \texttt{robots.txt}, yang dapat digunakan untuk menentukan bagian situs mana yang boleh dan tidak boleh diakses oleh \textit{crawler}. Pada sistem pemeriksa tautan rusak, aturan ini tidak dijadikan batasan mutlak karena tujuan utama adalah memastikan semua tautan dapat diperiksa. Namun, untuk menjaga sikap yang baik terhadap pemilik situs, keberadaan \texttt{robots.txt} tetap dapat dipertimbangkan sebagai acuan tambahan.  

Aspek lain yang penting adalah identitas \texttt{User-Agent}. Setiap permintaan HTTP yang dikirimkan sistem akan dilengkapi dengan informasi ini agar server mengetahui perangkat lunak apa yang sedang melakukan \textit{crawling}. Identitas tersebut sebaiknya mencantumkan nama aplikasi dan informasi kontak, sehingga administrator situs dapat mengenali sumber permintaan dengan jelas.  

Selain itu, sistem tidak dirancang untuk mengakses area privat, melewati autentikasi, atau mengambil konten yang bersifat berbayar. Aktivitas \textit{crawling} difokuskan pada tautan yang memang dapat diakses secara publik, sehingga tidak melanggar aturan kepemilikan maupun hak akses. Untuk mencegah server terbebani, pengaturan batas waktu dan laju permintaan juga digunakan. Dengan cara ini, sistem tetap menjalankan prinsip etika dasar dalam \textit{crawling} sekaligus mencapai tujuannya dalam memeriksa ketersediaan tautan.


\subsubsection*{Ekstraksi Tautan}
Subbab~\ref{subsec:0224-elemen-yang-mengandung-url} mencatat berbagai elemen HTML yang memiliki atribut berisi URL. Dalam konteks aplikasi pemeriksa tautan rusak, tidak semua elemen tersebut relevan. Analisis ini menentukan elemen mana yang diekstrak, atribut apa yang digunakan, serta alasan pemilihannya. Elemen-elemen yang dipilih adalah sebagai berikut:

\begin{itemize}
  \item \texttt{<a>}: menggunakan atribut \texttt{href} sebagai tautan utama antarhalaman. Elemen ini menjadi sumber navigasi paling penting sehingga wajib diperiksa.
  \item \texttt{<area>}: menggunakan atribut \texttt{href} pada \textit{image map}. Meskipun jarang dipakai, elemen ini tetap berfungsi sebagai tautan dan perlu diperiksa.
  \item \texttt{<link>}: menggunakan atribut \texttt{href} untuk menghubungkan dokumen dengan stylesheet, ikon, atau sumber daya eksternal lain. Jika rusak, tampilan halaman dapat terganggu.
  \item \texttt{<script>}: atribut \texttt{src} menunjuk ke berkas JavaScript eksternal. Tautan ini diperiksa karena jika rusak, fungsi interaktif halaman tidak berjalan.
  \item \texttt{<img>}: atribut \texttt{src} memuat lokasi gambar. Tautan rusak membuat gambar gagal ditampilkan sehingga perlu dicek.
  \item \texttt{<iframe>}: atribut \texttt{src} digunakan untuk menyematkan halaman lain di dalam halaman utama. Jika rusak, konten yang diembed tidak muncul.
  \item \texttt{<embed>}: menggunakan atribut \texttt{src} untuk konten eksternal seperti multimedia. Rusak berarti konten tidak dapat dimuat.
  \item \texttt{<object>}: atribut \texttt{data} menunjuk ke objek eksternal seperti PDF. Karena sering dipakai pada situs institusi, tautan ini harus diperiksa.
  \item \texttt{<source>}: atribut \texttt{src} digunakan dalam elemen \texttt{<audio>} atau \texttt{<video>} sebagai alternatif sumber media. Jika rusak, pemutaran media gagal.
  \item \texttt{<track>}: atribut \texttt{src} menyediakan berkas teks untuk subtitle atau caption. Rusak berarti fitur aksesibilitas tidak berfungsi.
  \item \texttt{<audio>}: atribut \texttt{src} menunjuk ke berkas audio. Jika rusak, konten audio tidak dapat diputar.
  \item \texttt{<video>}: atribut \texttt{src} menunjuk ke berkas video. Jika rusak, konten video gagal ditampilkan.
  \item \texttt{<input type="image">}: atribut \texttt{src} menunjuk ke gambar tombol kirim. Jika rusak, tombol tidak muncul di antarmuka.
\end{itemize}



\section{HTML~\cite{powell:10:htmlcss}}
\label{sec:02-html}

Hypertext Markup Language (HTML) merupakan bahasa markup standar yang digunakan untuk menyusun dan menampilkan halaman web pada \textit{browser}. HTML didefinisikan sebagai himpunan elemen yang dituliskan dalam bentuk tag, di mana setiap elemen dapat memiliki atribut untuk memberikan informasi tambahan. Dokumen HTML tersusun secara hierarkis dan secara konseptual dipandang sebagai sebuah pohon struktur yang dikenal dengan \textit{Document Object Model} (DOM).

Sejak diperkenalkan pada awal 1990-an, HTML telah berkembang melalui beberapa versi. HTML 4.01 yang dirilis pada tahun 1999 menjadi salah satu versi yang banyak digunakan dan bertahan cukup lama. Setelah itu muncul XHTML sebagai reformulasi HTML dalam sintaks XML, namun penerapannya terbatas. Versi terbaru adalah HTML5 yang dikembangkan oleh Web Hypertext Application Technology Working Group (WHATWG) dan kemudian diadopsi oleh World Wide Web Consortium (W3C), dengan dukungan yang lebih baik terhadap elemen semantik, multimedia, serta penulisan sintaks yang lebih sederhana.

\subsection{Struktur Dasar HTML}
\label{subsec:0224-struktur-dasar-html}

Sebuah dokumen HTML terdiri dari beberapa bagian utama. Dokumen diawali dengan deklarasi \texttt{<!DOCTYPE html>} yang memberi tahu \textit{browser} mengenai standar yang digunakan. Seluruh isi dokumen dibungkus dalam elemen \texttt{<html>} yang menjadi akar dari semua elemen lain.  

Di dalam \texttt{<html>} terdapat dua bagian penting, yaitu \texttt{<head>} dan \texttt{<body>}. Bagian \texttt{<head>} memuat informasi tentang dokumen seperti judul, metadata, dan pemanggilan sumber daya eksternal. Bagian \texttt{<body>} berisi konten utama yang ditampilkan kepada pengguna, seperti teks, gambar, tautan, tabel, maupun elemen multimedia.  


\begin{lstlisting}[language=HTML, caption={Struktur dasar dokumen HTML}, label={lst:html-basic-structure}]
<!DOCTYPE html>
<html>
<head>
    <title>Contoh Halaman</title>
    <meta charset="UTF-8">
    <link rel="stylesheet" href="style.css">
    <script src="script.js"></script>
</head>
<body>
    <h1>Selamat Datang</h1>
    <p>Ini adalah contoh struktur dasar HTML.</p>
</body>
</html>
\end{lstlisting}

Kode~\ref{lst:html-basic-structure} menunjukan struktur dasar dari HTML, baris pertama berisi deklarasi \texttt{<!DOCTYPE html>}. Elemen \texttt{<html>} membungkus seluruh dokumen, sedangkan \texttt{<head>} berisi \texttt{<title>} sebagai judul halaman, \texttt{<meta charset="UTF-8">} untuk menetapkan karakter encoding, \texttt{<link>} untuk memanggil stylesheet eksternal, serta \texttt{<script>} untuk menyertakan berkas JavaScript. Bagian \texttt{<body>} menampilkan konten kepada pengguna, dalam contoh ini berupa judul dan paragraf.

\subsection{Elemen Semantik dan Non-Semantik}
\label{subsec:0224-elemen-semantik-dan-non-semantik}

HTML menyediakan dua jenis elemen utama, yaitu elemen semantik dan elemen non-semantik.

\begin{enumerate}
    \item \textbf{Elemen Semantik} \\
    Elemen semantik adalah elemen yang memiliki makna jelas bagi \textit{browser} maupun pembaca, karena nama elemen menggambarkan fungsinya. Contoh elemen semantik antara lain:
    \begin{itemize}
        \item \texttt{<header>}: untuk bagian kepala suatu halaman atau bagian.
        \item \texttt{<nav>}: untuk kumpulan tautan navigasi.
        \item \texttt{<section>}: untuk sebuah bagian tematik dalam dokumen.
        \item \texttt{<article>}: untuk konten yang berdiri sendiri, seperti artikel berita.
        \item \texttt{<aside>}: untuk konten samping, seperti catatan atau iklan.
        \item \texttt{<footer>}: untuk bagian kaki halaman.
    \end{itemize}

    \item \textbf{Elemen Non-Semantik} \\
    Elemen non-semantik adalah elemen yang tidak menggambarkan makna spesifik dari kontennya, melainkan digunakan untuk keperluan pemformatan atau pengelompokan. Contoh elemen non-semantik adalah:
    \begin{itemize}
        \item \texttt{<div>}: untuk pengelompokan blok konten.
        \item \texttt{<span>}: untuk pengelompokan teks dalam baris.
    \end{itemize}
\end{enumerate}  

\subsection{Atribut Global dan Spesifik}
\label{subsec:0224-atribut-global-dan-spesifik}

Setiap elemen HTML dapat memiliki atribut yang berfungsi untuk memberikan informasi tambahan atau mengatur perilaku elemen tersebut. Atribut terbagi menjadi dua kategori, yaitu atribut global dan atribut spesifik.

Atribut global dapat digunakan pada hampir semua elemen HTML. Atribut ini bersifat umum karena tidak terikat pada fungsi tertentu dari elemen. Contoh atribut global meliputi:
\begin{itemize}
    \item \texttt{id} : identifikasi unik untuk sebuah elemen.
    \item \texttt{class} : pengelompokan elemen dengan gaya atau fungsi yang sama.
    \item \texttt{style} : mendefinisikan gaya inline menggunakan CSS.
    \item \texttt{title} : menyediakan keterangan tambahan yang biasanya ditampilkan sebagai \textit{tooltip}.
\end{itemize}

Atribut spesifik adalah atribut yang hanya berlaku pada elemen tertentu sesuai dengan fungsinya. Tabel~\ref{tab:html-specific-attributes} menunjukkan beberapa contoh atribut spesifik beserta elemen tempat atribut tersebut digunakan.

\begin{table}[H]
\centering
\caption{Contoh atribut spesifik pada elemen HTML}
\label{tab:html-specific-attributes}
\begin{tabular}{|l|l|p{7cm}|}
\hline
\textbf{Elemen} & \textbf{Atribut} & \textbf{Keterangan} \\ \hline
\texttt{<a>} & \texttt{href} & Menentukan alamat tujuan tautan. \\ \hline
\texttt{<img>} & \texttt{src} & Menentukan lokasi berkas gambar. \\ \hline
\texttt{<form>} & \texttt{action} & Menentukan alamat tujuan pengiriman data formulir. \\ \hline
\texttt{<script>} & \texttt{src} & Menentukan sumber berkas JavaScript eksternal. \\ \hline
\texttt{<link>} & \texttt{href} & Menentukan lokasi stylesheet atau sumber daya terkait lainnya. \\ \hline
\end{tabular}
\end{table}


% \subsection{Elemen yang Mengandung URL}
% \label{subsec:0224-elemen-yang-mengandung-url}

% Beberapa elemen HTML memiliki atribut URL yang menghubungkan dokumen dengan sumber daya lain. Elemen-elemen ini berperan penting dalam membangun keterhubungan antarhalaman maupun dengan berkas eksternal. Tabel~\ref{tab:html-url-elements} menampilkan beberapa elemen tersebut.

% \begin{center}
% \begin{longtable}{|l|l|p{5cm}|}
% \caption{Elemen HTML yang mengandung atribut URL} \label{tab:html-url-elements} \\

% \hline \multicolumn{1}{|c|}{\textbf{Elemen}} & \multicolumn{1}{c|}{\textbf{Atribut}} & \multicolumn{1}{c|}{\textbf{Keterangan}} \\ \hline 
% \endfirsthead

% \multicolumn{3}{c}%
% {{\bfseries \tablename\ \thetable{} -- lanjutan dari halaman sebelumnya}} \\
% \hline \multicolumn{1}{|c|}{\textbf{Elemen}} & \multicolumn{1}{c|}{\textbf{Atribut}} & \multicolumn{1}{c|}{\textbf{Keterangan}} \\ \hline 
% \endhead

% \hline \multicolumn{3}{|r|}{{Bersambung ke halaman berikutnya}} \\ \hline
% \endfoot

% \hline \hline
% \endlastfoot

% \texttt{<a>} & \texttt{href} & Menentukan alamat tujuan tautan. \\ \hline
% \texttt{<area>} & \texttt{href} & Menentukan alamat tujuan pada \textit{image map}. \\ \hline
% \texttt{<link>} & \texttt{href} & Menghubungkan dokumen dengan sumber daya eksternal, seperti stylesheet atau ikon. \\ \hline
% \texttt{<script>} & \texttt{src} & Menentukan lokasi berkas JavaScript eksternal. \\ \hline
% \texttt{<img>} & \texttt{src} & Menentukan lokasi berkas gambar. \\ \hline
% \texttt{<iframe>} & \texttt{src} & Menyematkan halaman web lain dalam bingkai. \\ \hline
% \texttt{<frame>} & \texttt{src} & Menentukan sumber halaman yang ditampilkan dalam sebuah frame. \\ \hline
% \texttt{<embed>} & \texttt{src} & Menyematkan konten eksternal, seperti multimedia. \\ \hline
% \texttt{<object>} & \texttt{data} & Menyematkan objek eksternal, seperti PDF atau aplikasi kecil. \\ \hline
% \texttt{<source>} & \texttt{src} & Menentukan sumber alternatif untuk elemen \texttt{<audio>} atau \texttt{<video>}. \\ \hline
% \texttt{<track>} & \texttt{src} & Menentukan lokasi berkas teks untuk \textit{caption} atau \textit{subtitle}. \\ \hline
% \texttt{<audio>} & \texttt{src} & Menentukan sumber berkas audio. \\ \hline
% \texttt{<video>} & \texttt{src} & Menentukan sumber berkas video. \\ \hline
% \texttt{<form>} & \texttt{action} & Menentukan alamat tujuan pengiriman data formulir. \\ \hline
% \texttt{<input type="image">} & \texttt{src} & Menentukan lokasi berkas gambar untuk tombol kirim. \\ \hline
% \texttt{<button>} & \texttt{formaction} & Menentukan alamat tujuan pengiriman data formulir khusus untuk tombol tersebut. \\ \hline
% \texttt{<base>} & \texttt{href} & Menentukan URL dasar untuk semua URL relatif dalam dokumen. \\ \hline
% \texttt{<meta>} & \texttt{http-equiv="refresh"} & Dapat memuat URL untuk mengarahkan ulang halaman. \\ \hline
% \texttt{<ins>} & \texttt{cite} & Menentukan URL sumber untuk penambahan teks. \\ \hline
% \texttt{<del>} & \texttt{cite} & Menentukan URL sumber untuk penghapusan teks. \\ \hline
% \texttt{<q>} & \texttt{cite} & Menentukan URL sumber untuk kutipan singkat. \\ \hline
% \texttt{<blockquote>} & \texttt{cite} & Menentukan URL sumber untuk kutipan panjang. \\ \hline
% \texttt{<isindex>} & \texttt{action} & Menentukan alamat tujuan query pencarian (elemen usang). \\ \hline

% \end{longtable}
% \end{center}



\section{JavaFX~\cite{sharan:22:learnjavafx17}}
\label{sec:02-javafx}
% \subsection{JavaFX}
% \label{subsec:0304-javafx}

Pemilihan JavaFX sebagai pustaka antarmuka pengguna didasarkan pada kebutuhan sistem yang harus disajikan dalam bentuk aplikasi desktop dengan tampilan sederhana, tabel yang mudah dibaca, serta kontrol proses yang jelas (lihat Subsubbab~\ref{subsec:0303-kebutuhan-non-fungsional}). Sebagaimana dijelaskan pada Subbab~\ref{sec:02-javafx}, JavaFX menawarkan paradigma \textit{scene graph} yang memungkinkan penyusunan antarmuka secara hierarkis dan konsisten. Paradigma ini memudahkan pengelolaan komponen visual yang diperlukan untuk menampilkan hasil pemeriksaan tautan secara \textit{streaming} (lihat Subsubbab~\ref{subsec:0303-kebutuhan-fungsional}).

Dalam implementasi, JavaFX akan digunakan melalui beberapa komponen utama berikut:

\begin{itemize}
  \item \textbf{Stage, Scene, dan Node}. Struktur dasar aplikasi akan mengikuti siklus hidup JavaFX dengan menurunkan kelas dari \texttt{Application}. Objek \texttt{Stage} akan berperan sebagai jendela utama, sedangkan \texttt{Scene} digunakan untuk menampung keseluruhan antarmuka. Komponen UI seperti \texttt{Button}, \texttt{Label}, dan \texttt{TextField} direpresentasikan sebagai turunan \texttt{Node} yang diorganisasi dalam kontainer tata letak, misalnya \texttt{VBox} atau \texttt{BorderPane}.

  \item \textbf{FXML dan Controller}. Untuk memisahkan logika aplikasi dari tampilan, struktur antarmuka didefinisikan secara deklaratif dalam berkas FXML. Atribut \texttt{fx:id} akan dipakai agar komponen UI dapat diakses dari kelas controller, sedangkan event handler seperti \texttt{onAction} digunakan untuk menangani interaksi pengguna, misalnya saat memulai atau menghentikan proses pemeriksaan tautan.

  \item \textbf{Komponen Tabel}. Hasil pemeriksaan tautan akan ditampilkan dalam bentuk tabel menggunakan \texttt{TableView}. Komponen ini mendukung penyajian data terstruktur dalam baris dan kolom, sehingga cocok untuk menampilkan daftar halaman yang diperiksa maupun daftar tautan rusak. Setiap kolom akan diikat dengan \texttt{Property} pada model data agar perubahan nilai dapat langsung tercermin pada tampilan.

  \item \textbf{Property dan Binding}. Untuk mendukung pembaruan data secara langsung, mekanisme \texttt{StringProperty}, \texttt{BooleanProperty}, dan \texttt{IntegerProperty} akan digunakan. Binding dua arah dimanfaatkan agar nilai pada komponen input dan model selalu konsisten, sedangkan binding satu arah memastikan perubahan status pemeriksaan langsung diperlihatkan pada label atau tabel.

  \item \textbf{Pengendalian Thread}. Karena proses pemeriksaan tautan berjalan secara paralel, pembaruan antarmuka pengguna harus dijalankan melalui \texttt{Platform.runLater()}. Hal ini menjamin sinkronisasi antara \textit{thread} pemeriksaan dengan \textit{thread} JavaFX, sehingga tampilan dapat diperbarui secara aman tanpa menimbulkan error \texttt{Not on FX application thread}.
\end{itemize}




\section{Jsoup}
\label{sec:02-jsoup}
% \subsection{Jsoup}
% \label{subsec:0304-jsoup}

Pemilihan Jsoup sebagai pustaka utama untuk pengambilan dan pemrosesan dokumen HTML didasarkan pada kebutuhan sistem untuk mengekstraksi tautan dari halaman web dengan cara yang andal, termasuk dari dokumen yang tidak valid atau tidak sepenuhnya sesuai standar. Kebutuhan ini secara eksplisit dinyatakan pada aspek ketahanan terhadap HTML tidak valid (lihat Subsubbab~\ref{subsec:0303-kebutuhan-non-fungsional}). Sebagaimana dijelaskan pada Subbab~\ref{sec:02-jsoup}, Jsoup menyediakan parser yang toleran kesalahan dan mendukung spesifikasi HTML5, sehingga struktur DOM tetap dapat dibentuk meskipun dokumen bermasalah. 

Dari sisi fungsional, penggunaan Jsoup relevan dengan kebutuhan untuk:
\begin{itemize}
  \item menerima masukan URL dan mengambil halaman web terkait (lihat Subsubbab~\ref{subsec:0303-kebutuhan-fungsional}).
  \item mengekstraksi seluruh tautan yang terdapat di dalam elemen HTML, khususnya elemen \texttt{<a>} dengan atribut \texttt{href}.
  \item menampilkan hasil pemeriksaan secara \textit{streaming}, sehingga setiap tautan yang diperoleh dari hasil parsing dapat segera diperiksa dan ditampilkan ke antarmuka pengguna.
\end{itemize}

Dalam implementasi, beberapa kelas dan method dari Jsoup akan digunakan secara langsung, yaitu:

\begin{itemize}
  \item \texttt{Jsoup}\\
  Kelas ini dipakai sebagai titik masuk untuk membuat koneksi HTTP ke sebuah halaman melalui \texttt{connect(String url)}. Method ini dilanjutkan dengan konfigurasi \texttt{userAgent(String ua)} untuk memenuhi etika \textit{crawling} dan \texttt{timeout(int millis)} untuk memenuhi kebutuhan non-fungsional terkait batas waktu respon. Eksekusi permintaan dilakukan dengan \texttt{get()} atau \texttt{execute()}, menghasilkan \texttt{Document} atau \texttt{Connection.Response}.

  \item \texttt{Connection} dan \texttt{Connection.Response}\\
  \texttt{Connection} digunakan untuk mengatur parameter koneksi, sementara \texttt{Connection.Response} diperlukan untuk membaca kode status HTTP dengan \texttt{statusCode()} serta metadata lain seperti \texttt{headers()}. Informasi ini mendukung kebutuhan untuk melabeli hasil pemeriksaan dengan status yang jelas (lihat Subsubbab~\ref{subsec:0303-kebutuhan-non-fungsional}).

  \item \texttt{Document}\\
  Objek ini merepresentasikan halaman web yang berhasil diambil. Method \texttt{select(String cssQuery)} akan dipakai untuk menemukan seluruh elemen \texttt{<a>} dengan atribut \texttt{href}. Dari sini dihasilkan objek \texttt{Elements}.

  \item \texttt{Elements} dan \texttt{Element}\\
  Koleksi \texttt{Elements} berisi banyak \texttt{Element} yang masing-masing mewakili sebuah tautan. Method penting yang digunakan adalah \texttt{attr("href")} untuk mengambil nilai atribut asli, \texttt{text()} untuk isi teks tautan, serta \texttt{absUrl("href")} untuk memperoleh URL absolut berdasarkan \texttt{baseUri()} dokumen. Normalisasi URL ini mendukung konsistensi identifikasi sumber daya (lihat Subsubbab~\ref{subsec:0303-kebutuhan-non-fungsional}).

\end{itemize}



\section{Java HTTP Client API}
\label{sec:02-java-http-client-api}
Java HTTP Client API secara garis besar adalah sebuah API dari Java yang digunakan untuk mengirim \textit{request} dan menerima \textit{response} melalui protokol HTTP. API ini pertama kali diperkenalkan pada Java 9 dengan nama \texttt{jdk.incubator.httpclient} sebagai \textit{incubating module}, yaitu modul percobaan yang digunakan untuk memperkenalkan API baru sebelum akhirnya ditetapkan sebagai bagian resmi dari Java 11. Java HTTP Client API mendukung penggunaan protokol HTTP/1.1 dan HTTP/2. Secara bawaan, pemilihan versi protokol dilakukan secara otomatis, di mana \texttt{HttpClient} akan mencoba menggunakan HTTP/2 terlebih dahulu dan melakukan \textit{fallback} ke HTTP/1.1 apabila server tidak mendukung HTTP/2. Selain mekanisme bawaan tersebut, pengembang juga dapat menetapkan versi protokol secara eksplisit melalui metode yang tersedia pada kelas \texttt{HttpClient}.

Selain mendukung dua versi protokol HTTP, API ini juga menyediakan dua model komunikasi, yaitu sinkron dan asinkron. Komunikasi sinkron berarti eksekusi program akan menunggu hingga \textit{response} (\textit{response}) dari server diterima sepenuhnya sebelum melanjutkan instruksi berikutnya. Sebaliknya, komunikasi asinkron menggunakan kelas \texttt{CompletableFuture}, yang memungkinkan hasil komputasi diperoleh di masa mendatang tanpa harus menunggu proses selesai. Objek \texttt{HttpClient} dalam API ini juga memiliki karakteristik penting untuk penggunaan di lingkungan \textit{multithreaded}. Objek ini bersifat \textit{immutable}, artinya konfigurasi tidak dapat diubah setelah dibuat, serta bersifat \textit{thread-safe}, artinya dapat diakses secara bersamaan oleh beberapa \textit{thread} tanpa menimbulkan \textit{race condition}.

\subsection{\texttt{HttpClient}}
\label{subsec:0228-httpclient}

Kelas \texttt{HttpClient} merupakan komponen inti dalam Java HTTP Client API yang digunakan untuk mengirim \textit{request} HTTP dan menerima \textit{response} dari server. Objek dari kelas ini dapat dibuat dengan menggunakan \textit{static method} \texttt{newHttpClient} dengan konfigurasi bawaan yang siap pakai, atau bisa menggunakan \textit{static method} \texttt{newBuilder} apabila ingin membuat objek dengan konfigurasi yang disesuaikan.

Berikut ini adalah beberapa \textit{method} untuk menetapkan konfigurasi, yang tersedia pada objek \textit{builder} \texttt{newBuilder}:

\begin{itemize}
    \item \texttt{version}\\
    \textit{Method} ini berfungsi untuk menentukan versi protokol HTTP yang akan digunakan, yaitu \texttt{HTTP\_1\_1} atau \texttt{HTTP\_2}. Nilai yang diberikan berasal dari enum \texttt{HttpClient.Version}. Dengan pengaturan ini, klien dapat diarahkan untuk selalu memakai versi tertentu atau menyesuaikan melalui negosiasi otomatis dengan server. Pengaturan versi yang jelas membantu menjaga konsistensi komunikasi dan mencegah terjadinya ketidakcocokan protokol antara klien dan server.
    
    \item \texttt{connectTimeout}\\
    \textit{Method} ini menerima sebuah objek \texttt{Duration} yang menyatakan batas waktu maksimal dalam membangun koneksi ke server. Jika dalam jangka waktu tersebut koneksi tidak berhasil dibuat, maka proses akan dihentikan dan menghasilkan \textit{exception}. Dengan demikian, aplikasi tidak akan menggantung terlalu lama ketika mencoba mengakses server yang tidak responsif.
    
    \item \texttt{cookieHandler}\\
    \textit{Method} ini digunakan untuk menetapkan objek \texttt{CookieHandler} yang akan menangani manajemen \textit{cookie} selama komunikasi antara \textit{client} dan \textit{server}. Ketika sebuah \textit{request} atau \textit{response} berisi \textit{cookie}, objek \texttt{CookieHandler} akan menyimpan, memperbarui, dan mengirimkan kembali \textit{cookie} tersebut sesuai aturan yang ditentukan. Dengan pengaturan ini, \textit{client} dapat mempertahankan sesi secara otomatis, mirip dengan perilaku \textit{browser} yang menyimpan \textit{cookie} dan menggunakannya kembali pada \textit{request} berikutnya.

    \item \texttt{followRedirects}\\
    \textit{Method} ini mengatur kebijakan ketika server memberikan instruksi \textit{redirect}. Parameter yang digunakan adalah nilai dari enum \texttt{HttpClient.Redirect}, seperti \texttt{ALWAYS} untuk selalu mengikuti \textit{redirect}, \texttt{NEVER} untuk tidak pernah mengikuti, dan \texttt{NORMAL} untuk selalu mengikuti \textit{redirect}, kecuali dari URL dengan \textit{scheme} HTTPS ke HTTP.
    
    \item \texttt{authenticator}\\
    \textit{Method} ini digunakan untuk menetapkan sebuah objek \texttt{Authenticator} yang akan dipanggil ketika server meminta autentikasi. Misalnya, saat server meminta kredensial melalui HTTP Basic Authentication, objek ini dapat menyediakan username dan password. Dengan demikian, akses ke sumber daya yang dilindungi dapat dilakukan secara otomatis tanpa perlu intervensi manual.
    
    \item \texttt{build}\\
    Setelah seluruh konfigurasi ditentukan, \textit{method} ini dipanggil untuk menghasilkan sebuah objek \texttt{HttpClient} yang siap digunakan. Objek ini bersifat \textit{immutable}, artinya setelah dibangun, konfigurasi tidak dapat diubah lagi, sehingga memastikan konsistensi perilaku selama penggunaan.
\end{itemize}

Selain \textit{static method}, kelas \texttt{HttpClient} juga menyediakan sejumlah \textit{instance method} yang digunakan secara langsung pada objek hasil \textit{build}. Salah satu yang paling penting adalah \texttt{send}, yaitu \textit{method} yang menjalankan sebuah \textit{request} HTTP secara sinkron. \textit{Method} ini menerima parameter berupa objek \texttt{HttpRequest} dan \texttt{HttpResponse.BodyHandler}, kemudian mengeksekusi \textit{request} hingga selesai dan mengembalikan objek \texttt{HttpResponse} yang berisi \textit{status code}, \textit{header}, serta isi \textit{response body}. Selain itu, terdapat juga \textit{method} \texttt{sendAsync} yang bekerja secara asinkron. Sama seperti \texttt{send}, \textit{method} ini menerima objek \texttt{HttpRequest} dan \texttt{HttpResponse.BodyHandler}, namun alih-alih langsung menghasilkan \texttt{HttpResponse}, \textit{method} ini mengembalikan sebuah \texttt{CompletableFuture} yang merepresentasikan hasil \textit{response} yang akan tersedia di kemudian waktu.



\subsection{\texttt{HttpRequest}}
\label{subsec:0228-httprequest}


Kelas \texttt{HttpRequest} merepresentasikan sebuah \textit{request} HTTP yang akan dikirim menggunakan \texttt{HttpClient}. Objek dari kelas ini bersifat \textit{immutable}, sehingga setiap konfigurasi harus ditentukan terlebih dahulu melalui \texttt{HttpRequest.Builder} sebelum \textit{request} dibangun. Setelah dibuat, konfigurasi sebuah \texttt{HttpRequest} tidak dapat diubah lagi.

Untuk membangun sebuah \texttt{HttpRequest}, dapat digunakan \textit{static method} \texttt{newBuilder} yang menghasilkan objek \texttt{HttpRequest.Builder}. Pada \textit{builder} ini tersedia sejumlah \textit{method} yang digunakan untuk menetapkan konfigurasi permintaan:

\begin{itemize}
    \item \texttt{uri}\\
    \textit{Method} ini digunakan untuk menentukan alamat tujuan dari \textit{request} HTTP dalam bentuk objek \texttt{java.net.URI}. URI bersifat wajib karena menentukan ke mana \textit{request} akan dikirim.
    
    \item \texttt{header} dan \texttt{headers}\\
    \textit{Method} \texttt{header} menetapkan satu pasangan \textit{name-value} sebagai \textit{header}, sedangkan \texttt{headers} digunakan untuk menetapkan beberapa pasangan sekaligus. Header merupakan informasi tambahan yang menyertai permintaan, seperti \texttt{Content-Type} atau \texttt{Authorization}.
    
    \item \texttt{timeout}\\
    \textit{Method} ini menerima parameter berupa objek \texttt{Duration} untuk menentukan batas waktu maksimal pemrosesan permintaan. Jika respons tidak diterima dalam waktu yang ditetapkan, maka akan terjadi \textit{timeout}.
    
    \item \texttt{version}\\
    \textit{Method} ini digunakan untuk menentukan versi protokol HTTP yang akan digunakan pada \textit{request} tertentu. Nilai yang diberikan berupa enum \texttt{HttpClient.Version}.
    
    \item \texttt{expectContinue}\\
    \textit{Method} ini mengatur apakah \textit{request} menggunakan mekanisme \textit{Expect: 100-continue}. Jika diset true, client akan mengirim \textit{header} terlebih dahulu dan menunggu konfirmasi dari server sebelum mengirim isi \textit{request body}. Mekanisme ini berguna untuk \textit{request} dengan \textit{request body} berukuran besar.
    
    \item \texttt{GET}, \texttt{POST}, \texttt{PUT}, \texttt{DELETE}, dan \texttt{method}\\
    \textit{Methods} ini menentukan jenis operasi HTTP yang akan dilakukan. \texttt{GET} dan \texttt{DELETE} tidak menyertakan \textit{request body}, sementara \texttt{POST} dan \texttt{PUT} membutuhkan objek \texttt{BodyPublisher} untuk mendefinisikan data yang akan dikirim melalui \textit{request body}. \textit{Method} generik \texttt{method} memungkinkan penentuan jenis operasi lain seperti PATCH dan HEAD.

    
    \item \texttt{build}\\
    Setelah semua konfigurasi ditetapkan, \textit{method} ini dipanggil untuk menghasilkan sebuah objek \texttt{HttpRequest} yang siap digunakan oleh \texttt{HttpClient}.
\end{itemize}



\subsection{\texttt{HttpResponse}}
\label{subsec:0228-httpresponse}

Kelas \texttt{HttpResponse} merepresentasikan \textit{response} yang diterima setelah sebuah \textit{request} dieksekusi oleh \texttt{HttpClient}. Kelas ini menggunakan parameter generik \texttt{<T>} yang menunjukkan tipe data dari isi \textit{response body}. Nilai generik ini ditentukan oleh \texttt{HttpResponse.BodyHandler} yang digunakan saat mengirim permintaan.

Beberapa \textit{method} penting yang tersedia pada \texttt{HttpResponse} antara lain:

\begin{itemize}
    \item \texttt{statusCode}\\
    Mengembalikan nilai berupa bilangan bulat yang merepresentasikan HTTP \textit{status code} dari \textit{response}, misalnya \texttt{200} untuk \textit{OK} atau \texttt{404} untuk \textit{Not Found}.
    
    \item \texttt{headers}\\
    Mengembalikan objek \texttt{HttpHeaders} yang berisi seluruh \textit{header} dari \textit{response}. Setiap \textit{header} dapat diakses berdasarkan nama dan dapat memiliki lebih dari satu nilai.
    
    \item \texttt{body}\\
    Mengembalikan isi \textit{response body} dengan tipe data sesuai parameter generik \texttt{<T>}. Contohnya, jika menggunakan \texttt{BodyHandler<String>}, maka \textit{response body} akan dikembalikan dalam bentuk string.
    
    \item \texttt{previousResponse}\\
    Mengembalikan objek \texttt{Optional<HttpResponse<T>>} yang berisi \textit{response} sebelumnya apabila terjadi \textit{redirect}. Jika tidak ada, nilai yang dikembalikan adalah kosong.
    
    \item \texttt{sslSession}\\
    Mengembalikan informasi sesi SSL dalam bentuk \texttt{Optional<SSLSession>} jika koneksi dilakukan melalui protokol HTTPS.
    
    \item \texttt{uri}\\
    Mengembalikan objek \texttt{URI} yang merepresentasikan alamat tujuan akhir dari permintaan, termasuk setelah terjadi \textit{redirect}.
    
    \item \texttt{version}\\
    Mengembalikan versi protokol HTTP yang digunakan pada komunikasi, berupa nilai dari enum \texttt{HttpClient.Version}.
    
    \item \texttt{request}\\
    Mengembalikan objek \texttt{HttpRequest} yang digunakan untuk menghasilkan \textit{response} ini.
\end{itemize}



\subsection{\texttt{HttpHeaders}}
\label{subsec:0228-httpheaders}

Kelas \texttt{HttpHeaders} merepresentasikan kumpulan \textit{header} yang terdapat pada \textit{request} maupun \textit{response} HTTP. Objek dari kelas ini bersifat \textit{immutable}, sehingga nilai yang tersimpan tidak dapat diubah setelah dibuat. Setiap \textit{header} dapat memiliki lebih dari satu nilai, dan semua nilai disimpan dalam struktur yang mempertahankan urutan kemunculannya.

Beberapa \textit{method} penting yang tersedia pada \texttt{HttpHeaders} antara lain:

\begin{itemize}
    \item \texttt{firstValue}\\
    Mengembalikan nilai pertama dari \textit{header} dengan nama tertentu dalam bentuk \textit{string}. \textit{Method} ini berguna ketika hanya satu nilai yang relevan dari sebuah \textit{header} yang mungkin memiliki banyak nilai.
    
    \item \texttt{allValues}\\
    Mengembalikan seluruh nilai dari \textit{header} dengan nama tertentu dalam bentuk daftar. \textit{Method} ini digunakan ketika sebuah \textit{header} dapat berisi lebih dari satu nilai.
    
    \item \texttt{map}\\
    Mengembalikan seluruh isi \textit{header} dalam bentuk struktur \texttt{Map<String, List<String>>}, dengan setiap kunci berupa nama \textit{header} dan nilainya berupa daftar nilai terkait. Dengan cara ini, semua \textit{header} dan nilainya dapat diakses secara langsung.
\end{itemize}


\subsection{Contoh Kode Program}
\label{subsec:0228-contoh-kode-program}

Kode~\ref{lst:httpclient-example} menunjukkan contoh penggunaan Java HTTP Client API untuk melakukan permintaan HTTP ke sebuah alamat web dan menampilkan hasil tanggapannya. Kode ini ditulis untuk menggambarkan bagaimana kelas-kelas utama yang telah dijelaskan sebelumnya, yaitu \texttt{HttpClient}, \texttt{HttpRequest}, \texttt{HttpResponse}, dan \texttt{HttpHeaders}, digunakan secara bersama-sama dalam sebuah program nyata. Hasil eksekusi dari kode ini berupa informasi \textit{status code}, daftar \textit{header}, serta isi \textit{response body} yang diterima dari server tujuan.

\begin{lstlisting}[language=Java, caption={Contoh penggunaan Java HTTP Client API}, label={lst:httpclient-example}]
public class HttpClientExample {
    public static void main(String[] args) throws Exception {
        // Membangun HttpClient dengan konfigurasi yang relevan
        HttpClient client = HttpClient.newBuilder()
                .version(HttpClient.Version.HTTP_2)
                .connectTimeout(Duration.ofSeconds(5))
                .followRedirects(HttpClient.Redirect.NORMAL)
                .build();

        // Menyusun HttpRequest: URI, header, timeout, dan method GET
        HttpRequest request = HttpRequest.newBuilder(URI.create("https://informatika.unpar.ac.id"))
                .header("User-Agent", "BrokenLinkChecker 1.0")
                .timeout(Duration.ofSeconds(10))
                .GET()
                .build();

        HttpResponse<String> response = client.send(
                request, HttpResponse.BodyHandlers.ofString()
        );

        System.out.println("Status Code : " + response.statusCode());

        HttpHeaders headers = response.headers();
        Map<String, List<String>> headerMap = headers.map();
        for (Map.Entry<String, List<String>> e : headerMap.entrySet()) {
            String name = e.getKey();
            String values = String.join(", ", e.getValue());
            System.out.println(name + ": " + values);
        }

        System.out.println("Body        : " + response.body());
    }
}
\end{lstlisting}

Alur dari Kode~\ref{lst:httpclient-example} dimulai dengan pembuatan objek \texttt{HttpClient} melalui \texttt{newBuilder()} dan penetapan beberapa konfigurasi, yaitu versi protokol HTTP melalui \texttt{version()}, batas waktu koneksi dengan \texttt{connectTimeout()}, serta kebijakan \textit{redirect} menggunakan \texttt{followRedirects()}. Setelah itu dipanggil \texttt{build()} untuk menghasilkan objek \texttt{HttpClient}. Berikutnya, sebuah \texttt{HttpRequest} dibangun dengan \texttt{HttpRequest.newBuilder(URI)} untuk menentukan alamat tujuan, kemudian ditambahkan informasi tambahan melalui \texttt{header()}, ditetapkan batas waktu menggunakan \texttt{timeout()}, dan jenis operasi HTTP dipilih dengan \texttt{GET()}, lalu diselesaikan dengan \texttt{build()}. Permintaan tersebut dikirim menggunakan \texttt{HttpClient.send()} dengan parameter \texttt{HttpResponse.BodyHandlers.ofString()}, sehingga isi \textit{response body} diproses menjadi \textit{string}. Hasil eksekusi berupa objek \texttt{HttpResponse} yang menyediakan \textit{status code}, kumpulan \textit{header}, serta isi \textit{response body}. Daftar \textit{header} diperoleh dari \texttt{HttpHeaders.map()}, kemudian ditampilkan seluruhnya dalam bentuk pasangan \textit{name-value}, diikuti dengan pencetakan isi \textit{response body}.



