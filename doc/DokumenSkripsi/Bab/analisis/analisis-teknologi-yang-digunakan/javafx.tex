% \subsection{JavaFX}
% \label{subsec:0304-javafx}

Pemilihan JavaFX sebagai pustaka antarmuka pengguna didasarkan pada kebutuhan sistem yang harus disajikan dalam bentuk aplikasi desktop dengan tampilan sederhana, tabel yang mudah dibaca, serta kontrol proses yang jelas (lihat Subsubbab~\ref{subsec:0303-kebutuhan-non-fungsional}). Sebagaimana dijelaskan pada Subbab~\ref{sec:02-javafx}, JavaFX menawarkan paradigma \textit{scene graph} yang memungkinkan penyusunan antarmuka secara hierarkis dan konsisten. Paradigma ini memudahkan pengelolaan komponen visual yang diperlukan untuk menampilkan hasil pemeriksaan tautan secara \textit{streaming} (lihat Subsubbab~\ref{subsec:0303-kebutuhan-fungsional}).

Dalam implementasi, JavaFX akan digunakan melalui beberapa komponen utama berikut:

\begin{itemize}
  \item \textbf{Stage, Scene, dan Node}. Struktur dasar aplikasi akan mengikuti siklus hidup JavaFX dengan menurunkan kelas dari \texttt{Application}. Objek \texttt{Stage} akan berperan sebagai jendela utama, sedangkan \texttt{Scene} digunakan untuk menampung keseluruhan antarmuka. Komponen UI seperti \texttt{Button}, \texttt{Label}, dan \texttt{TextField} direpresentasikan sebagai turunan \texttt{Node} yang diorganisasi dalam kontainer tata letak, misalnya \texttt{VBox} atau \texttt{BorderPane}.

  \item \textbf{FXML dan Controller}. Untuk memisahkan logika aplikasi dari tampilan, struktur antarmuka didefinisikan secara deklaratif dalam berkas FXML. Atribut \texttt{fx:id} akan dipakai agar komponen UI dapat diakses dari kelas controller, sedangkan event handler seperti \texttt{onAction} digunakan untuk menangani interaksi pengguna, misalnya saat memulai atau menghentikan proses pemeriksaan tautan.

  \item \textbf{Komponen Tabel}. Hasil pemeriksaan tautan akan ditampilkan dalam bentuk tabel menggunakan \texttt{TableView}. Komponen ini mendukung penyajian data terstruktur dalam baris dan kolom, sehingga cocok untuk menampilkan daftar halaman yang diperiksa maupun daftar tautan rusak. Setiap kolom akan diikat dengan \texttt{Property} pada model data agar perubahan nilai dapat langsung tercermin pada tampilan.

  \item \textbf{Property dan Binding}. Untuk mendukung pembaruan data secara langsung, mekanisme \texttt{StringProperty}, \texttt{BooleanProperty}, dan \texttt{IntegerProperty} akan digunakan. Binding dua arah dimanfaatkan agar nilai pada komponen input dan model selalu konsisten, sedangkan binding satu arah memastikan perubahan status pemeriksaan langsung diperlihatkan pada label atau tabel.

  \item \textbf{Pengendalian Thread}. Karena proses pemeriksaan tautan berjalan secara paralel, pembaruan antarmuka pengguna harus dijalankan melalui \texttt{Platform.runLater()}. Hal ini menjamin sinkronisasi antara \textit{thread} pemeriksaan dengan \textit{thread} JavaFX, sehingga tampilan dapat diperbarui secara aman tanpa menimbulkan error \texttt{Not on FX application thread}.
\end{itemize}

