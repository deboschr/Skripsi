
Hypertext Markup Language (HTML) merupakan bahasa markup standar yang digunakan untuk menyusun dan menampilkan halaman web pada \textit{browser}. HTML didefinisikan sebagai himpunan elemen yang dituliskan dalam bentuk tag, di mana setiap elemen dapat memiliki atribut untuk memberikan informasi tambahan. Dokumen HTML tersusun secara hierarkis dan secara konseptual dipandang sebagai sebuah pohon struktur yang dikenal dengan \textit{Document Object Model} (DOM).

Sejak diperkenalkan pada awal 1990-an, HTML telah berkembang melalui beberapa versi. HTML 4.01 yang dirilis pada tahun 1999 menjadi salah satu versi yang banyak digunakan dan bertahan cukup lama. Setelah itu muncul XHTML sebagai reformulasi HTML dalam sintaks XML, namun penerapannya terbatas. Versi terbaru adalah HTML5 yang dikembangkan oleh Web Hypertext Application Technology Working Group (WHATWG) dan kemudian diadopsi oleh World Wide Web Consortium (W3C), dengan dukungan yang lebih baik terhadap elemen semantik, multimedia, serta penulisan sintaks yang lebih sederhana.

\subsection{Struktur Dasar HTML}
\label{subsec:0224-struktur-dasar-html}

Sebuah dokumen HTML terdiri dari beberapa bagian utama. Dokumen diawali dengan deklarasi \texttt{<!DOCTYPE html>} yang memberi tahu \textit{browser} mengenai standar yang digunakan. Seluruh isi dokumen dibungkus dalam elemen \texttt{<html>} yang menjadi akar dari semua elemen lain.  

Di dalam \texttt{<html>} terdapat dua bagian penting, yaitu \texttt{<head>} dan \texttt{<body>}. Bagian \texttt{<head>} memuat informasi tentang dokumen seperti judul, metadata, dan pemanggilan sumber daya eksternal. Bagian \texttt{<body>} berisi konten utama yang ditampilkan kepada pengguna, seperti teks, gambar, tautan, tabel, maupun elemen multimedia.  


\begin{lstlisting}[language=HTML, caption={Struktur dasar dokumen HTML}, label={lst:html-basic-structure}]
<!DOCTYPE html>
<html>
<head>
    <title>Contoh Halaman</title>
    <meta charset="UTF-8">
    <link rel="stylesheet" href="style.css">
    <script src="script.js"></script>
</head>
<body>
    <h1>Selamat Datang</h1>
    <p>Ini adalah contoh struktur dasar HTML.</p>
</body>
</html>
\end{lstlisting}

Kode~\ref{lst:html-basic-structure} menunjukan struktur dasar dari HTML, baris pertama berisi deklarasi \texttt{<!DOCTYPE html>}. Elemen \texttt{<html>} membungkus seluruh dokumen, sedangkan \texttt{<head>} berisi \texttt{<title>} sebagai judul halaman, \texttt{<meta charset="UTF-8">} untuk menetapkan karakter encoding, \texttt{<link>} untuk memanggil stylesheet eksternal, serta \texttt{<script>} untuk menyertakan berkas JavaScript. Bagian \texttt{<body>} menampilkan konten kepada pengguna, dalam contoh ini berupa judul dan paragraf.

\subsection{Elemen Semantik dan Non-Semantik}
\label{subsec:0224-elemen-semantik-dan-non-semantik}

HTML menyediakan dua jenis elemen utama, yaitu elemen semantik dan elemen non-semantik.

\begin{enumerate}
    \item \textbf{Elemen Semantik} \\
    Elemen semantik adalah elemen yang memiliki makna jelas bagi \textit{browser} maupun pembaca, karena nama elemen menggambarkan fungsinya. Contoh elemen semantik antara lain:
    \begin{itemize}
        \item \texttt{<header>}: untuk bagian kepala suatu halaman atau bagian.
        \item \texttt{<nav>}: untuk kumpulan tautan navigasi.
        \item \texttt{<section>}: untuk sebuah bagian tematik dalam dokumen.
        \item \texttt{<article>}: untuk konten yang berdiri sendiri, seperti artikel berita.
        \item \texttt{<aside>}: untuk konten samping, seperti catatan atau iklan.
        \item \texttt{<footer>}: untuk bagian kaki halaman.
    \end{itemize}

    \item \textbf{Elemen Non-Semantik} \\
    Elemen non-semantik adalah elemen yang tidak menggambarkan makna spesifik dari kontennya, melainkan digunakan untuk keperluan pemformatan atau pengelompokan. Contoh elemen non-semantik adalah:
    \begin{itemize}
        \item \texttt{<div>}: untuk pengelompokan blok konten.
        \item \texttt{<span>}: untuk pengelompokan teks dalam baris.
    \end{itemize}
\end{enumerate}  

\subsection{Atribut Global dan Spesifik}
\label{subsec:0224-atribut-global-dan-spesifik}

Setiap elemen HTML dapat memiliki atribut yang berfungsi untuk memberikan informasi tambahan atau mengatur perilaku elemen tersebut. Atribut terbagi menjadi dua kategori, yaitu atribut global dan atribut spesifik.

Atribut global dapat digunakan pada hampir semua elemen HTML. Atribut ini bersifat umum karena tidak terikat pada fungsi tertentu dari elemen. Contoh atribut global meliputi:
\begin{itemize}
    \item \texttt{id} : identifikasi unik untuk sebuah elemen.
    \item \texttt{class} : pengelompokan elemen dengan gaya atau fungsi yang sama.
    \item \texttt{style} : mendefinisikan gaya inline menggunakan CSS.
    \item \texttt{title} : menyediakan keterangan tambahan yang biasanya ditampilkan sebagai \textit{tooltip}.
\end{itemize}

Atribut spesifik adalah atribut yang hanya berlaku pada elemen tertentu sesuai dengan fungsinya. Tabel~\ref{tab:html-specific-attributes} menunjukkan beberapa contoh atribut spesifik beserta elemen tempat atribut tersebut digunakan.

\begin{table}[H]
\centering
\caption{Contoh atribut spesifik pada elemen HTML}
\label{tab:html-specific-attributes}
\begin{tabular}{|l|l|p{7cm}|}
\hline
\textbf{Elemen} & \textbf{Atribut} & \textbf{Keterangan} \\ \hline
\texttt{<a>} & \texttt{href} & Menentukan alamat tujuan tautan. \\ \hline
\texttt{<img>} & \texttt{src} & Menentukan lokasi berkas gambar. \\ \hline
\texttt{<form>} & \texttt{action} & Menentukan alamat tujuan pengiriman data formulir. \\ \hline
\texttt{<script>} & \texttt{src} & Menentukan sumber berkas JavaScript eksternal. \\ \hline
\texttt{<link>} & \texttt{href} & Menentukan lokasi stylesheet atau sumber daya terkait lainnya. \\ \hline
\end{tabular}
\end{table}


% \subsection{Elemen yang Mengandung URL}
% \label{subsec:0224-elemen-yang-mengandung-url}

% Beberapa elemen HTML memiliki atribut URL yang menghubungkan dokumen dengan sumber daya lain. Elemen-elemen ini berperan penting dalam membangun keterhubungan antarhalaman maupun dengan berkas eksternal. Tabel~\ref{tab:html-url-elements} menampilkan beberapa elemen tersebut.

% \begin{center}
% \begin{longtable}{|l|l|p{5cm}|}
% \caption{Elemen HTML yang mengandung atribut URL} \label{tab:html-url-elements} \\

% \hline \multicolumn{1}{|c|}{\textbf{Elemen}} & \multicolumn{1}{c|}{\textbf{Atribut}} & \multicolumn{1}{c|}{\textbf{Keterangan}} \\ \hline 
% \endfirsthead

% \multicolumn{3}{c}%
% {{\bfseries \tablename\ \thetable{} -- lanjutan dari halaman sebelumnya}} \\
% \hline \multicolumn{1}{|c|}{\textbf{Elemen}} & \multicolumn{1}{c|}{\textbf{Atribut}} & \multicolumn{1}{c|}{\textbf{Keterangan}} \\ \hline 
% \endhead

% \hline \multicolumn{3}{|r|}{{Bersambung ke halaman berikutnya}} \\ \hline
% \endfoot

% \hline \hline
% \endlastfoot

% \texttt{<a>} & \texttt{href} & Menentukan alamat tujuan tautan. \\ \hline
% \texttt{<area>} & \texttt{href} & Menentukan alamat tujuan pada \textit{image map}. \\ \hline
% \texttt{<link>} & \texttt{href} & Menghubungkan dokumen dengan sumber daya eksternal, seperti stylesheet atau ikon. \\ \hline
% \texttt{<script>} & \texttt{src} & Menentukan lokasi berkas JavaScript eksternal. \\ \hline
% \texttt{<img>} & \texttt{src} & Menentukan lokasi berkas gambar. \\ \hline
% \texttt{<iframe>} & \texttt{src} & Menyematkan halaman web lain dalam bingkai. \\ \hline
% \texttt{<frame>} & \texttt{src} & Menentukan sumber halaman yang ditampilkan dalam sebuah frame. \\ \hline
% \texttt{<embed>} & \texttt{src} & Menyematkan konten eksternal, seperti multimedia. \\ \hline
% \texttt{<object>} & \texttt{data} & Menyematkan objek eksternal, seperti PDF atau aplikasi kecil. \\ \hline
% \texttt{<source>} & \texttt{src} & Menentukan sumber alternatif untuk elemen \texttt{<audio>} atau \texttt{<video>}. \\ \hline
% \texttt{<track>} & \texttt{src} & Menentukan lokasi berkas teks untuk \textit{caption} atau \textit{subtitle}. \\ \hline
% \texttt{<audio>} & \texttt{src} & Menentukan sumber berkas audio. \\ \hline
% \texttt{<video>} & \texttt{src} & Menentukan sumber berkas video. \\ \hline
% \texttt{<form>} & \texttt{action} & Menentukan alamat tujuan pengiriman data formulir. \\ \hline
% \texttt{<input type="image">} & \texttt{src} & Menentukan lokasi berkas gambar untuk tombol kirim. \\ \hline
% \texttt{<button>} & \texttt{formaction} & Menentukan alamat tujuan pengiriman data formulir khusus untuk tombol tersebut. \\ \hline
% \texttt{<base>} & \texttt{href} & Menentukan URL dasar untuk semua URL relatif dalam dokumen. \\ \hline
% \texttt{<meta>} & \texttt{http-equiv="refresh"} & Dapat memuat URL untuk mengarahkan ulang halaman. \\ \hline
% \texttt{<ins>} & \texttt{cite} & Menentukan URL sumber untuk penambahan teks. \\ \hline
% \texttt{<del>} & \texttt{cite} & Menentukan URL sumber untuk penghapusan teks. \\ \hline
% \texttt{<q>} & \texttt{cite} & Menentukan URL sumber untuk kutipan singkat. \\ \hline
% \texttt{<blockquote>} & \texttt{cite} & Menentukan URL sumber untuk kutipan panjang. \\ \hline
% \texttt{<isindex>} & \texttt{action} & Menentukan alamat tujuan query pencarian (elemen usang). \\ \hline

% \end{longtable}
% \end{center}
