\textit{Uniform Resource Identifier} (URI) adalah string karakter yang digunakan untuk mengidentifikasi suatu sumber daya. Identifikasi tersebut dapat berupa lokasi, nama, atau kombinasi keduanya. Spesifikasi URI ditetapkan dalam RFC~3986 yang mendefinisikan aturan sintaks, komponen, serta mekanisme representasi sumber daya.

RFC~3986 mendeskripsikan URI sebagai konsep umum yang memiliki beberapa bentuk. Dua bentuk yang paling dikenal adalah \textit{Uniform Resource Locator} (URL) dan \textit{Uniform Resource Name} (URN). URL adalah URI yang menyertakan informasi lokasi dan mekanisme akses terhadap sumber daya, contohnya \texttt{http://unpar.ac.id/index.html}. URN adalah URI yang berfungsi sebagai nama tetap suatu sumber daya tanpa menyertakan lokasi aksesnya, contohnya \texttt{urn:isbn:0451450523}. RFC juga membedakan URI yang bersifat hierarkis dan URI yang bersifat opak. URI hierarkis memiliki komponen seperti \textit{authority} dan \textit{path}, misalnya skema \texttt{http}. URI opak tidak dapat diuraikan ke dalam komponen hierarkis, misalnya skema \texttt{mailto}.

\subsection{Struktur Sintaks URL}
\label{subsec:0202-struktur-url}

URL, sebagai bentuk URI yang bersifat hierarkis, memiliki struktur sintaks yang terdiri atas beberapa komponen. Format umum penulisan URL ditunjukkan sebagai berikut:
\begin{center}
\texttt{scheme:[//[user:password@]host[:port]]path[?query][\#fragment]}
\end{center}

Komponen-komponen dalam URL dijelaskan sebagai berikut:
\begin{itemize}
  \item \textbf{Scheme}\\
  Bagian ini dituliskan pada awal URL dan diakhiri dengan tanda titik dua (\texttt{:}). Scheme menunjukkan protokol atau mekanisme akses yang digunakan, contohnya \texttt{http}, \texttt{https}, \texttt{ftp}, atau \texttt{mailto}.
  
  \item \textbf{Authority}\\
  Bagian ini bersifat opsional dan diawali dengan dua garis miring (\texttt{//}). Authority dapat terdiri atas tiga subkomponen, yaitu \textit{userinfo}, \textit{host}, dan \textit{port}. \textit{Userinfo} berisi informasi pengguna dengan format \texttt{user:password@} yang digunakan dalam beberapa skema seperti FTP. \textit{Host} adalah lokasi sumber daya yang dapat berupa nama domain atau alamat IP. \textit{Port} adalah nomor port koneksi, misalnya \texttt{80} untuk HTTP atau \texttt{443} untuk HTTPS. Jika port tidak dituliskan, maka \textit{port default} dari skema tersebut digunakan.
  
  \item \textbf{Path}\\
  Bagian ini menunjukkan jalur menuju sumber daya pada host. Path terdiri atas segmen-segmen yang dipisahkan tanda garis miring (\texttt{/}). Di dalamnya dapat terdapat \textit{dot-segments}, yaitu segmen khusus berupa \texttt{.} yang merujuk ke direktori saat ini dan \texttt{..} yang merujuk ke direktori induk.
  
  \item \textbf{Query}\\
  Bagian ini bersifat opsional dan diawali dengan tanda tanya (\texttt{?}). Query berisi parameter dalam format pasangan \texttt{key=value}. Jika terdapat lebih dari satu parameter, maka masing-masing dipisahkan dengan tanda \texttt{\&}.
  
  \item \textbf{Fragment}\\
  Bagian ini bersifat opsional dan diawali dengan tanda pagar (\texttt{\#}). Fragment digunakan untuk merujuk ke bagian tertentu dari sumber daya. Informasi ini tidak dikirim ke server, melainkan diproses oleh agen pengguna seperti browser.
\end{itemize}

\subsection{Kategori Karakter dan \textit{Percent-Encoding}}
\label{subsec:0202-karakter-percent-encoding}

RFC~3986 menetapkan kategori karakter yang dapat digunakan dalam URI, beserta aturan pengkodeannya.
\begin{itemize}
  \item \textbf{Unreserved characters}\\
  Karakter ini dapat digunakan langsung tanpa pengkodean tambahan. Termasuk di dalamnya huruf alfabet A sampai Z dan a sampai z, digit angka 0 sampai 9, serta simbol \texttt{-}, \texttt{\_}, \texttt{.}, dan \texttt{\textasciitilde}.
  
  \item \textbf{Reserved characters}\\
  Karakter ini memiliki fungsi khusus tergantung pada konteks penggunaannya. RFC membagi reserved characters menjadi dua kelompok. General delimiters meliputi \texttt{:}, \texttt{/}, \texttt{?}, \texttt{\#}, \texttt{[}, \texttt{]}, dan \texttt{@}. Subcomponent delimiters meliputi \texttt{!}, \texttt{\$}, \texttt{\&}, \texttt{'}, \texttt{(}, \texttt{)}, \texttt{*}, \texttt{+}, \texttt{,}, \texttt{;}, dan \texttt{=}. Jika karakter ini digunakan di luar konteks yang diperbolehkan, maka harus dikodekan.
  
  \item \textbf{Karakter lain}\\
  Karakter non-ASCII, spasi, dan simbol lain yang tidak termasuk dalam kategori di atas tidak dapat digunakan secara langsung. Karakter ini harus dikodekan sebelum dapat dimasukkan ke dalam URI.
\end{itemize}

Pengkodean karakter dilakukan dengan mekanisme \textit{percent-encoding}. Mekanisme ini menggantikan karakter dengan representasi heksadesimalnya dalam format \texttt{\%HH}, di mana \texttt{HH} adalah nilai ASCII karakter tersebut. Sebagai contoh, spasi ditulis sebagai \texttt{\%20}, tanda kutip ganda sebagai \texttt{\%22}, dan tanda pagar sebagai \texttt{\%23}. RFC juga mengatur bahwa karakter unreserved yang tidak perlu dikodekan sebaiknya dituliskan dalam bentuk aslinya. Selain itu, digit heksadesimal dalam \textit{percent-encoding} harus ditulis dengan huruf besar.

\subsection{Referensi Absolut dan Relatif}
\label{subsec:0202-referensi-url}

URL dapat berupa referensi absolut maupun relatif. URL absolut mencantumkan seluruh komponen utama termasuk scheme dan authority, sehingga dapat diinterpretasikan secara mandiri. Contohnya adalah \texttt{https://unpar.ac.id/page.html}. URL relatif tidak mencantumkan scheme atau authority dan hanya menyertakan path, query, atau fragment. Contohnya adalah \texttt{images/logo.png}. Selain itu terdapat pula same-document reference, yaitu URL yang hanya berisi fragment untuk merujuk ke bagian tertentu dalam dokumen yang sama, contohnya \texttt{\#section2}.

RFC~3986 mendefinisikan mekanisme resolusi yang digunakan untuk menyusun URL absolut dari sebuah referensi relatif. Resolusi dilakukan dengan menentukan base URI, menggabungkannya dengan URL relatif, lalu menyederhanakan hasilnya. Salah satu tahap penting dalam resolusi adalah penghapusan dot-segments. Prosedur ini menghilangkan segmen \texttt{.} dan \texttt{..} agar path hasil resolusi berada dalam bentuk yang ringkas dan tidak ambigu.
