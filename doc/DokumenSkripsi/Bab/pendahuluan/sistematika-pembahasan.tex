Sistematika pembahasan dalam penelitian ini disusun sebagai berikut:

\begin{enumerate}
    \item Bab 1 Pendahuluan \\
    Bab ini berisi latar belakang dilakukannya penelitian, rumusan masalah, tujuan penelitian, batasan masalah, metodologi yang digunakan, serta sistematika pembahasan.

    \item Bab 2 Landasan Teori \\
    Bab ini membahas teori-teori yang menjadi dasar dalam pengembangan perangkat lunak pemeriksa tautan rusak pada situs web. Pembahasan mencakup protokol HTTP, URI, konsep dasar situs web, \textit{web crawling}, serta pustaka dan teknologi yang digunakan seperti Jsoup, Java \texttt{HttpClient}, JavaFX dan Gradle.

    \item Bab 3 Analisis \\
    Bab ini membahas analisis permasalahan pemeriksaan tautan rusak pada situs web, peninjauan perangkat lunak serupa, perumusan kebutuhan fungsional dan non-fungsional, serta analisis teknologi yang digunakan dalam pengembangan aplikasi.

    \item Bab 4 Perancangan \\
    Bab ini membahas perancangan perangkat lunak berdasarkan hasil analisis yang sudah dibuat. Perancangan meliputi perancangan kelas dan perancangan antarmuka pengguna.

    \item Bab 5 Implementasi dan Pengujian \\
    Bab ini memaparkan hasil implementasi perangkat lunak pemeriksa tautan rusak berdasarkan perancangan yang sudah dibuat. Selain itu, bab ini juga memaparkan hasil pengujian fungsional dan eksperimental serta analisis terhadap hasil pengujian.
    
    \item Bab 6 Kesimpulan dan Saran \\
    Bab ini berisi kesimpulan dari hasil pengembangan perangkat lunak pemeriksa tautan rusak pada situs web, serta saran pengembangan lebih lanjut yang didasarkan pada hasil analisis pengujian yang sudah dilakukan.
\end{enumerate}