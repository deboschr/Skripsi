Tujuan dari mempelajari \textit{library node-schedule} adalah untuk mendukung pengembangan dari modul sinkronisator data, sehingga sinkronisasi data dapat dilakukan secara otomatis. Sumber informasi dalam mempelajari \textit{library} adalah dokumentasi \textit{node-schedule} yang ada di \textit{website} resmi NPM dan \textit{Repository GitHub} resmi dari \textit{node-schedule}.

Versi \textit{node-schedule} yang digunakan dalam eksplorasi adalah \textit{node-module} versi 2.1.1
    
\begin{enumerate}[label*=\arabic*.,ref=\arabic*]
    \item Konsep Dasar\\
    \textit{Library node-schedule} adalah \textit{library Node.js} yang digunakan untuk melakukan penjadwalan eksekusi tugas (\textit{task scheduling}). Cara kerja dari \textit{node-schedule} adalah menggunakan pola waktu yang didefinisikan pengguna untuk mendaftarkan tugas ke\textit{ event loop Node.js}. \textit{Library} ini secara internal menggunakan \textit{timer} untuk memeriksa waktu dan menjalankan \textit{callback} ketika pola waktu terpenuhi.

    Berikut adalah tiga pendekatan penjadwalan yang didukung:
    \begin{enumerate}[label*=\arabic*.,ref=\arabic*]
        \item Penjadwalan dengan Waktu Spesifik\\
        Pendekatan ini memungkinkan penjadwalan tugas yang dijalankan pada waktu tertentu. Pengguna dapat menentukan tanggal dan waktu yang spesifik, biasanya menggunakan objek \texttt{Date} di JavaScript. Misalnya, sebuah tugas dapat dijadwalkan untuk dijalankan pada 2024-12-09 10:00:00.

        \item Penjadwalan dengan Waktu Berulang\\
        Pendekatan ini memungkinkan penjadwalan tugas yang dijalankan secara berulang menggunakan interval tertentu. Sebagai contoh, tugas dapat dijalankan setiap 5 menit atau setiap hari pada waktu yang sama. Penggunaan waktu berulang akan menggunakan objek \texttt{RecurrenceRule} yang dapat diatur untuk mendefinisikan pola pengulangan.
             
        \item Penjadwalan dengan Pola Cron\\
        Pendekatan ini memungkinkan penjadwalan tugas yang dijalankan secara berulang menggunakan pola \textit{cron}. Pola ini menggunakan format \textit{string} yang terdiri dari lima atau enam kolom (\texttt{* * * * * *}), yang masing-masing merepresentasikan detik (opsional), menit, jam, hari dalam bulan, bulan, dan hari dalam minggu. Sebagai contoh, pola \texttt{0 9 * * *} akan menjadwalkan tugas untuk berjalan setiap hari pada pukul 09:00.

    \end{enumerate}
        
    \item Instalasi\\
    Untuk mengunduh \textit{library} \texttt{node-schedule}, langkah-langkahnya adalah sebagai berikut:
    
    \begin{enumerate}[label*=\arabic*.,ref=\arabic*]
        \item Pastikan Node.js sudah diunduh.
        Periksa dengan perintah:
        \begin{verbatim}
        node -v
        npm -v
        \end{verbatim}

        \item Inisialisasi proyek Node.js (jika belum ada).
        \begin{verbatim}
        mkdir my-node-project
        cd my-node-project
        npm init -y
        \end{verbatim}

        \item Instal \textit{library} \texttt{node-schedule}.
        \begin{verbatim}
        npm install node-schedule --save
        \end{verbatim}

        \item Verifikasi instalasi.
        \begin{verbatim}
        npm list node-schedule
        \end{verbatim}
    \end{enumerate}
        
    \item API dan Contoh Penggunaan\\
    Secara fungsional, API pada \textit{node-schedule} dibagi menjadi tiga kategori utama:
    
    \begin{enumerate}[label*=\arabic*.,ref=\arabic*]
        \item Penjadwalan \textit{Job}\\
            
        \begin{enumerate}[label=\alph*.]
            \item \textbf{\textit{schedule.scheduleJob(nameOrRule, rule, callback)}}\\
            Berfungsi untuk menjadwalkan \textit{job}. Berikut adalah contoh untuk tiga pendekatan penjadwalan: \\
            \begin{itemize}
                \item Penjadwalan dengan Waktu Spesifik\\
                \textbf{Contoh:}
                \begin{lstlisting}[language=Javascript,caption={Penjadwalan dengan Waktu Spesifik}]
const date = new Date(2024, 11, 9, 10, 0, 0); // 9 Desember 2024, pukul 10:00:00
const job = schedule.scheduleJob('specificTimeJob', date, () => {
    console.log('Job executed at a spesific time:', new Date());
});
\end{lstlisting}
                \textbf{Hasil:} \textit{Job} dijalankan satu kali pada \texttt{2024-12-09 10:00:00}.

                \item Penjadwalan dengan Waktu Berulang\\
                \textbf{Contoh:}
                \vspace{-0.1cm}
                \begin{lstlisting}[language=Javascript,caption={Penjadwalan dengan Waktu Berulang}]
const rule = new schedule.RecurrenceRule();
rule.hour = 10; // Setiap jam 10 pagi
rule.minute = 0; // Pada menit 0
const job = schedule.scheduleJob('recurringJob', rule, () => {
    console.log('Recurring job executed at', new Date()); 
});
\end{lstlisting}
                \textbf{Hasil:} \textit{Job} dijalankan setiap hari pada pukul 10:00.

                \item Penjadwalan dengan Pola \textit{Cron}\\
                \textbf{Contoh:}
                \vspace{-0.1cm}
                \begin{lstlisting}[language=Javascript,caption={Penjadwalan dengan Pola Cron}]
const job = schedule.scheduleJob('cronPatternJob', '*/10 * * * * *', () => {
    console.log('Job executed based on cron pattern at', new Date());
});
\end{lstlisting}
                \textbf{Hasil:} \textit{Job} dijalankan setiap 10 detik sesuai pola \textit{cron} \texttt{*/10 * * * * *}.
            \end{itemize}
                
            \item \textit{\textbf{job.reschedule(rule)}}\\
                Berfungsi untuk menjadwalkan ulang \textit{job} yang sudah ada.\\
            \textbf{Contoh:}
            \vspace{-0.1cm}
            \begin{lstlisting}[language=Javascript,caption={Pendjawalan Ulang}]
job.reschedule('*/1 * * * *'); // Ubah menjadi setiap menit
\end{lstlisting}
            \textbf{Hasil:} \textit{`Job executed based on cron pattern at'} di \textit{console} setiap 1 menit setelah \textit{reschedule}.
                
        \end{enumerate}

        \item Pembatalan \textit{Job}\\
        \begin{enumerate}[label=\alph*.]
            \item \textbf{\textit{schedule.cancelJob(nameOrJob)}}\\
            Berfungsi untuk membatalkan \textit{job} berdasarkan nama atau referensi objek \textit{job}.\\
            \begin{itemize}
                \item Pembatalan dengan nama \textit{job}\\
                \textbf{Contoh:}
                \vspace{-0.1cm}
                \begin{lstlisting}[language=Javascript,caption={Pembatalan dengan nama Job}]
schedule.cancelJob('job1')
\end{lstlisting}
                \textbf{Hasil:} \textit{Job} dengan nama \textit{job1} dibatalkan

                \item Pembatalan dengan referensi objek \textit{job}\\
                \textbf{Contoh:}
                \vspace{-0.1cm}
                \begin{lstlisting}[language=Javascript,caption={Pembatalan dengan referensi Objek Job}]
schedule.cancelJob(job)
\end{lstlisting}
                \textbf{Hasil:} \textit{job} yang disimpan dalam variabel \textit{job} dibatalkan

            \end{itemize}
                
            \item \textbf{\textit{job.cancel()}}\\
            Berfungsi untuk membatalkan \textit{job} tertentu secara manual.\\
            \textbf{Contoh:}
            \vspace{-0.1cm}
            \begin{lstlisting}[language=Javascript,caption={Pembatalan job tertentu secara manual}]
job.cancel()
\end{lstlisting}
            \textbf{Hasil:} \textit{job} dibatalkan
            
            \item \textbf{\textit{job.cancelNext()}}\\
            Berfungsi untuk membatalkan eksekusi \textit{job} berikutnya pada \textit{job} yang berulang.\\
            \textbf{Contoh:}
            \vspace{-0.1cm}
            \begin{lstlisting}[language=Javascript,caption={Pembatalan eksekusi job berikutnya}]
job.cancelNext()
\end{lstlisting}
            \textbf{Hasil:} eksekusi \textit{job} berikutnya dibatalkan
                
        \end{enumerate}

        \item \textit{Debugging} dan Informasi \textit{Job}\\
        \begin{enumerate}[label=\alph*.]
            \item \textbf{\textit{schedule.scheduledJobs}}\\
            Menyediakan daftar semua \textit{job} yang sedang dijadwalkan dalam bentuk objek.\\
            \textbf{Contoh:}
            \begin{lstlisting}[language=Javascript,caption={Penyediaan Daftar Semua Job}]
Object.keys(schedule.scheduledJobs).forEach((jobName) => {
    console.log(jobName)
})
\end{lstlisting}
            \textbf{Hasil:} Mengambalikan nama job yang aktif
                
            \item \textbf{\textit{job.nextInvocation()}}\\
            Berfungsi untuk mengembalikan waktu eksekusi berikutnya dari \textit{job}.\\
            \textbf{Contoh:}
            \begin{lstlisting}[language=Javascript,caption={Pengembalian Waktu Eksekusi}]
job.nextInvocation()
\end{lstlisting}
            \textbf{Hasil:} Mengembalikan waktu eksekusi berikutnya 
                
            \item \textit{\textbf{job.invoke()}}\\
            Menjalankan \textit{job} secara manual tanpa menunggu waktu yang dijadwalkan.\\
            \textbf{Contoh:}
            \begin{lstlisting}[language=Javascript,caption={Menjalankan Job secara manual}]
job.invoke()
\end{lstlisting}
            \textbf{Hasil:} \textit{job} dieksekusi tanpa menunggu waktu eksekusi.
                
        \end{enumerate}

    \end{enumerate}
\end{enumerate}