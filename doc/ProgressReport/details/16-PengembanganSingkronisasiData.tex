Modul Sinkronisator Data dikembangkan untuk memastikan bahwa data antara basis data aplikasi web hasil \textit{generate} dan aplikasi internal tetap sinkron secara otomatis berdasarkan jadwal yang telah ditentukan. Modul ini dirancang dengan pendekatan modular dan fleksibel untuk mendukung kebutuhan sinkronisasi data yang kompleks.

\begin{enumerate}[label*=\arabic*.,ref=\arabic*]
    \item \textbf{Lingkungan Pengembangan}\\
    Modul ini dibangun menggunakan teknologi berikut:
    \begin{itemize}
        \item \texttt{axios}: Untuk melakukan HTTP \textit{/} ke API aplikasi internal.
        \item \texttt{node-schedule}: Untuk menjadwalkan sinkronisasi secara otomatis berdasarkan ekspresi \texttt{cron}.
    \end{itemize}

    \item \textbf{Implementasi Fitur}\\
    Implementasi modul mencakup dua fitur utama, yaitu penjadwalan sinkronisasi dan sinkronisasi data.
        \begin{enumerate}[label=\alph*.]
            \item \textbf{Penjadwalan Sinkronisasi}\\
            Penjadwalan sinkronisasi dilakukan menggunakan pustaka \texttt{node-schedule}. Setiap pekerjaan (\textit{job}) disimpan dalam objek konfigurasi yang mencakup nama, jadwal \texttt{cron}, dan detail sinkronisasi. Penjadwalan ini memastikan bahwa sinkronisasi berjalan secara otomatis sesuai dengan waktu yang telah ditentukan.

            Berikut adalah contoh implementasi penjadwalan sinkronisasi:
            \begin{lstlisting}[language=Javascript,caption={Penjadwalan Job}]
const createJob = async (dataJob) => {
	try {
		const job = schedule.scheduleJob(
			dataJob.name,
			dataJob.cron,
			async function () {
				try {
					await etl(dataJob);

					const currDate = new Date().toLocaleString();
					console.log(`Job ${dataJob.name} selesai pada ${currDate}`);
				} catch (etlError) {
					console.error(
						`Error dalam ETL untuk job ${dataJob.name}: ${etlError.message}`
					);
				}
			}
		);

		JOBS[dataJob.name] = job;
	} catch (error) {
		throw error;
	}
};
\end{lstlisting}

            \item \textbf{Sinkronisasi Data}\\
            Sinkronisasi data dilakukan dalam tiga tahap utama: \textit{Extract}, \textit{Transform}, dan \textit{Load} (ETL). Setiap tahap dirancang sebagai fungsi modular yang dapat digunakan kembali.

                \begin{itemize}
                    \item \textit{Extract Data}\\
                    Tahap ini bertugas mengambil data dari basis data aplikasi web hasil \textit{generate}. Koneksi ke basis data dilakukan menggunakan konfigurasi yang telah ditentukan. Data diekstrak dari tabel yang relevan sesuai kebutuhan sinkronisasi.

                    Berikut adalah implementasi fungsi \textit{Extract}:
                    \begin{lstlisting}[language=Javascript,caption={Extract Data}]
async function extract(dataJob) {
	try {
		console.log("Melakukan ekstraksi data...");

		const connection = await connectToDatabase(dataJob.source_db);

		const tables = dataJob.tables;
		const extractedData = {};

		for (const table of tables) {
			console.log(`Mengambil data dari tabel: ${table}`);
			extractedData[table] = await connection.query(`SELECT * FROM ${table}`);
		}

		return extractedData;
	} catch (error) {
		throw new Error(`Gagal mengekstrak data: ${error.message}`);
	}
}
\end{lstlisting}

                    \item \textit{Transform Data}\\
                    Data hasil ekstraksi diubah ke dalam format yang dapat diterima oleh API aplikasi internal. Proses transformasi dilakukan berdasarkan aturan (\textit{mapping rules}) yang didefinisikan dalam konfigurasi sinkronisasi.

                    Berikut adalah implementasi fungsi \textit{Transform}:
                    \begin{lstlisting}[language=Javascript,caption={Transform Data}]
async function transform(dataJob, data) {
	try {
		const transformRules = dataJob.transform;
		const transformedData = {};

		for (const [table, rows] of Object.entries(data)) {
			transformedData[table] = rows.map((row) => {
				const transformedRow = {};
				const rules = transformRules[table];

				for (const [sourceField, targetField] of Object.entries(rules)) {
					transformedRow[targetField] = row[sourceField];
				}

				return transformedRow;
			});
		}

		return transformedData;
	} catch (error) {
		throw new Error(`Gagal mentransformasi data: ${error.message}`);
	}
}
\end{lstlisting}

                    \item \textit{Load Data}\\
                    Data yang telah ditransformasi dikirimkan ke API aplikasi internal menggunakan pustaka \texttt{axios}. Proses ini mencakup validasi token autentikasi dan pengiriman data ke \textit{endpoint} yang ditentukan.

                    Berikut adalah implementasi fungsi \textit{Load}:
                    \begin{lstlisting}[language=Javascript,caption={Load Data}]
async function load(dataJob, data) {
	try {
		const endpoint = dataJob.endpoint;
		const headers = dataJob.headers;

		for (const [table, rows] of Object.entries(data)) {
			await axios.post(endpoint, rows, { headers });
		}
	} catch (error) {
		throw new Error(`Gagal memuat data ke API tujuan: ${error.message}`);
	}
}
\end{lstlisting}
                \end{itemize}
        \end{enumerate}
\end{enumerate}
