\begin{lstlisting}[language=Javascript,caption={HTML, CSS dan JavaScript}]
<!DOCTYPE html>
<html lang="en">
<head>
    <meta charset="UTF-8">
    <meta name="viewport" content="width=device-width, initial-scale=1.0">
    <title>Contoh Sederhana</title>
    <style>
        /* Menggabungkan CSS langsung di dalam tag <style> */
        body {
            font-family: Arial, sans-serif;
            padding: 20px;
            text-align: center;
        }

        p {
            color: #333;
        }

        button {
            padding: 10px 20px;
            font-size: 16px;
            cursor: pointer;
            background-color: #4CAF50;
            color: white;
            border: none;
            border-radius: 5px;
        }
    </style>
</head>
<body>
    <p id="message">Halo, ini adalah contoh penggunaan HTML, CSS, dan JavaScript!</p>
    <button onclick="changeColor()">Ubah Warna</button>

    <script>
        // Menggabungkan JavaScript langsung di dalam tag <script>
        function changeColor() {
            document.getElementById("message").style.color = "blue";
        }
    </script>
</body>
</html>
\end{lstlisting}