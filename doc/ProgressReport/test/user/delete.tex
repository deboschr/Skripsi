\begin{table}[H]
    \centering
    \begin{tabular}{|p{0.5cm}|p{3cm}|p{5cm}|p{5cm}|p{1.5cm}|}
        \hline
        \rowcolor[HTML]{DAE8FC} 
        \textbf{No} & \textbf{Skenario} & \textbf{Kasus Uji} & \textbf{Hasil yang Diharapkan} & \textbf{Hasil} \\ \hline
        1 & Menghapus pengguna dengan ID valid & 
        id: 1 & 
        Status code 200 OK, berhasil menghapus pengguna dengan ID 1 dan mengembalikan pesan sukses & 
        Berhasil \\ \hline
        2 & Mengakses endpoint dengan ID yang tidak ditemukan & 
        id: 9999 & 
        Status code 404 Not Found, gagal menghapus pengguna karena ID tidak ditemukan & 
        Berhasil \\ \hline
        3 & Tidak mengirimkan ID pada endpoint & 
        - & 
        Status code 400 Bad Request, gagal menghapus pengguna karena ID tidak diberikan & 
        Berhasil \\ \hline
        4 & Mengirim ID yang bukan angka & 
        id: "abc" & 
        Status code 400 Bad Request, gagal menghapus pengguna karena ID tidak valid & 
        Berhasil \\ \hline
        5 & Gagal menghapus pengguna karena memiliki relasi dengan tabel lain & 
        id: 2 & 
        Status code 409 Conflict, gagal menghapus pengguna karena pengguna berelasi dengan tabel lain (ON DELETE RESTRICT) & 
        Berhasil \\ \hline
    \end{tabular}
    \caption{Pengujian Fungsional Endpoint DELETE /user/:id}
    \label{tab:user_delete_testing}
\end{table}
