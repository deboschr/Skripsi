\begin{table}[H]
    \centering
    \begin{tabular}{|p{0.5cm}|p{3cm}|p{5cm}|p{5cm}|p{1.5cm}|}
        \hline
        \rowcolor[HTML]{DAE8FC} 
        \textbf{No} & \textbf{Skenario} & \textbf{Kasus Uji} & \textbf{Hasil yang Diharapkan} & \textbf{Hasil} \\ \hline
        1 & Mengirim data valid dengan ID yang ada & 
        id: 1 \newline host: localhost \newline port: 5433 & 
        Status code 200 OK, berhasil memperbarui data konfigurasi basis data & 
        Berhasil \\ \hline
        2 & Mengirim data valid dengan ID yang tidak ditemukan & 
        id: 999 \newline host: localhost \newline port: 5433 & 
        Status code 404 Not Found, gagal karena ID tidak ditemukan & 
        Berhasil \\ \hline
        3 & Mengirim data tanpa atribut apapun & 
        Tidak ada body request & 
        Status code 400 Bad Request, gagal karena tidak ada data untuk diperbarui & 
        Berhasil \\ \hline
    \end{tabular}
    \caption{Pengujian Fungsional Endpoint PATCH /db-config/:id}
    \label{tab:db_config_patch_testing}
\end{table}
