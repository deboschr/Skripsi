\begin{table}[H]
    \centering
    \begin{tabular}{|p{0.5cm}|p{3cm}|p{5cm}|p{5cm}|p{1.5cm}|}
        \hline
        \rowcolor[HTML]{DAE8FC} 
        \textbf{No} & \textbf{Skenario} & \textbf{Kasus Uji} & \textbf{Hasil yang Diharapkan} & \textbf{Hasil} \\ \hline
        1 & Mengirim data valid & 
        asset\_static\_id: 1 \newline placeholder\_values: \{"KEY": "value"\} & 
        Status code 201 Created, berhasil membuat aset dinamis baru & 
        Berhasil \\ \hline
        2 & Mengirim data tanpa \textit{asset\_static\_id} & 
        placeholder\_values: \{"KEY": "value"\} & 
        Status code 400 Bad Request, gagal karena \textit{asset\_static\_id} wajib dikirimkan & 
        Berhasil \\ \hline
        3 & Mengirim data tanpa \textit{placeholder\_values} & 
        asset\_static\_id: 1 & 
        Status code 400 Bad Request, gagal karena \textit{placeholder\_values} wajib dikirimkan & 
        Berhasil \\ \hline
        4 & Mengirim data dengan tipe data atribut tidak valid & 
        asset\_static\_id: "string" \newline placeholder\_values: \{"KEY": "value"\} & 
        Status code 400 Bad Request, gagal karena tipe data atribut tidak valid & 
        Berhasil \\ \hline
    \end{tabular}
    \caption{Pengujian Fungsional Endpoint POST /asset-dynamic}
    \label{tab:asset_dynamic_post_testing}
\end{table}
