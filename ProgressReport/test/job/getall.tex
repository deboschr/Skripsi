\begin{table}[H]
    \centering
    \begin{tabular}{|p{0.5cm}|p{3cm}|p{5cm}|p{5cm}|p{1.5cm}|}
        \hline
        \rowcolor[HTML]{DAE8FC} 
        \textbf{No} & \textbf{Skenario} & \textbf{Kasus Uji} & \textbf{Hasil yang Diharapkan} & \textbf{Hasil} \\ \hline
        1 & Mengakses semua data job tanpa pagination & 
        Tidak ada parameter & 
        Status code 200 OK, mengembalikan semua data job & 
        Berhasil \\ \hline
        2 & Mengakses data job dengan pagination & 
        page: 1 \newline limit: 10 & 
        Status code 200 OK, mengembalikan data job sesuai parameter pagination & 
        Berhasil \\ \hline
        3 & Mengirim parameter pagination tidak valid & 
        page: -1 \newline limit: 10 & 
        Status code 400 Bad Request, gagal karena parameter pagination tidak valid & 
        Berhasil \\ \hline
    \end{tabular}
    \caption{Pengujian Fungsional Endpoint GET /job}
    \label{tab:job_getall_testing}
\end{table}
