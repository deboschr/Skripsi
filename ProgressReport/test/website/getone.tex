\begin{table}[H]
    \centering
    \begin{tabular}{|p{0.5cm}|p{3cm}|p{5cm}|p{5cm}|p{1.5cm}|}
        \hline
        \rowcolor[HTML]{DAE8FC} 
        \textbf{No} & \textbf{Skenario} & \textbf{Kasus Uji} & \textbf{Hasil yang Diharapkan} & \textbf{Hasil} \\ \hline
        1 & Mengakses endpoint dengan ID valid & 
        id: 1 & 
        Status code 200 OK, mengembalikan detail website dengan ID 1 & 
        Berhasil \\ \hline
        2 & Mengakses endpoint dengan ID yang tidak ditemukan & 
        id: 9999 & 
        Status code 404 Not Found, gagal karena ID website tidak ditemukan & 
        Berhasil \\ \hline
        3 & Mengakses endpoint dengan ID tidak valid & 
        id: "abc" & 
        Status code 400 Bad Request, gagal karena format ID tidak valid & 
        Berhasil \\ \hline
    \end{tabular}
    \caption{Pengujian Fungsional Endpoint GET /website/:id}
    \label{tab:website_getone_testing}
\end{table}
