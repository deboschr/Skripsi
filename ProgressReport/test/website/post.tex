\begin{table}[H]
    \centering
    \begin{tabular}{|p{0.5cm}|p{3cm}|p{5cm}|p{5cm}|p{1.5cm}|}
        \hline
        \rowcolor[HTML]{DAE8FC} 
        \textbf{No} & \textbf{Skenario} & \textbf{Kasus Uji} & \textbf{Hasil yang Diharapkan} & \textbf{Hasil} \\ \hline
        1 & Mengirim data valid & 
        name: Website Target 1 \newline 
        dir\_path: ../website-result & 
        Status code 201 Created, berhasil membuat website baru dan mengembalikan data website & 
        Berhasil \\ \hline
        2 & Mengirim data dengan nama yang sudah digunakan & 
        name: Website Target 1 \newline 
        dir\_path: ../website-result & 
        Status code 409 Conflict, gagal karena nama website sudah ada & 
        Berhasil \\ \hline
        3 & Mengirim data dengan direktori kosong & 
        name: Website Target 2 & 
        Status code 400 Bad Request, gagal karena direktori tidak boleh kosong & 
        Berhasil \\ \hline
        4 & Mengirim data tanpa nama website & 
        dir\_path: ../website-result & 
        Status code 400 Bad Request, gagal karena nama website tidak boleh kosong & 
        Berhasil \\ \hline
    \end{tabular}
    \caption{Pengujian Fungsional Endpoint POST /website}
    \label{tab:website_create_testing}
\end{table}
