\begin{table}[H]
    \centering
    \begin{tabular}{|p{0.5cm}|p{3cm}|p{5cm}|p{5cm}|p{1.5cm}|}
        \hline
        \rowcolor[HTML]{DAE8FC} 
        \textbf{No} & \textbf{Skenario} & \textbf{Kasus Uji} & \textbf{Hasil yang Diharapkan} & \textbf{Hasil} \\ \hline
        1 & Mengirim data valid untuk memperbarui nama website & 
        id: 1 \newline 
        name: Website Target Updated & 
        Status code 200 OK, berhasil memperbarui nama website & 
        Berhasil \\ \hline
        2 & Mengirim data dengan nama website yang sudah digunakan & 
        id: 2 \newline 
        name: Website Target 1 & 
        Status code 409 Conflict, gagal karena nama website sudah digunakan & 
        Berhasil \\ \hline
        3 & Mengirim data dengan direktori kosong & 
        id: 1 \newline 
        dir\_path: & 
        Status code 400 Bad Request, gagal karena direktori tidak boleh kosong & 
        Berhasil \\ \hline
        4 & Mengirim data tanpa atribut apapun & 
        id: 1 & 
        Status code 400 Bad Request, gagal karena data yang dikirim tidak valid & 
        Berhasil \\ \hline
        5 & Mengirim data untuk ID yang tidak ditemukan & 
        id: 999 \newline 
        name: Website Target Updated & 
        Status code 404 Not Found, gagal karena ID website tidak ditemukan & 
        Berhasil \\ \hline
    \end{tabular}
    \caption{Pengujian Fungsional Endpoint PATCH /website/:id}
    \label{tab:website_update_testing}
\end{table}
