\begin{table}[H]
    \centering
    \begin{tabular}{|p{0.5cm}|p{3cm}|p{5cm}|p{5cm}|p{1.5cm}|}
        \hline
        \rowcolor[HTML]{DAE8FC} 
        \textbf{No} & \textbf{Skenario} & \textbf{Kasus Uji} & \textbf{Hasil yang Diharapkan} & \textbf{Hasil} \\ \hline
        1 & Mengirim data valid untuk folder & 
        website\_id: 1 \newline parent\_id: null \newline name: public \newline type: folder \newline content: null \newline placeholders: null & 
        Status code 201 Created, berhasil membuat folder di aset statis & 
        Berhasil \\ \hline
        2 & Mengirim data valid untuk file & 
        website\_id: 1 \newline parent\_id: 1 \newline name: apikey.json \newline type: file \newline content: eyJ7e0tF \newline placeholders: {"KEY\_NAME": "secret\_key"} & 
        Status code 201 Created, berhasil membuat file di aset statis & 
        Berhasil \\ \hline
        3 & Mengirim data tanpa atribut wajib & 
        website\_id: 1 \newline type: file & 
        Status code 400 Bad Request, gagal karena atribut wajib tidak lengkap & 
        Berhasil \\ \hline
        4 & Mengirim data dengan tipe data atribut tidak valid & 
        website\_id: "string" \newline name: public & 
        Status code 400 Bad Request, gagal karena tipe data atribut tidak valid & 
        Berhasil \\ \hline
    \end{tabular}
    \caption{Pengujian Fungsional Endpoint POST /asset-static}
    \label{tab:asset_static_post_testing}
\end{table}
