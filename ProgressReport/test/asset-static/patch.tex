\begin{table}[H]
    \centering
    \begin{tabular}{|p{0.5cm}|p{3cm}|p{5cm}|p{5cm}|p{1.5cm}|}
        \hline
        \rowcolor[HTML]{DAE8FC} 
        \textbf{No} & \textbf{Skenario} & \textbf{Kasus Uji} & \textbf{Hasil yang Diharapkan} & \textbf{Hasil} \\ \hline
        1 & Mengirim data valid untuk memperbarui folder & 
        id: 1 \newline name: assets \newline type: folder & 
        Status code 200 OK, berhasil memperbarui data aset statis & 
        Berhasil \\ \hline
        2 & Mengirim data valid untuk memperbarui file & 
        id: 2 \newline name: apikey.json \newline placeholders: {"KEY\_NAME": "new\_secret"} & 
        Status code 200 OK, berhasil memperbarui data aset statis & 
        Berhasil \\ \hline
        3 & Mengirim data tanpa atribut apapun & 
        Tidak ada body request & 
        Status code 400 Bad Request, gagal karena tidak ada data untuk diperbarui & 
        Berhasil \\ \hline
    \end{tabular}
    \caption{Pengujian Fungsional Endpoint PATCH /asset-static/:id}
    \label{tab:asset_static_patch_testing}
\end{table}
