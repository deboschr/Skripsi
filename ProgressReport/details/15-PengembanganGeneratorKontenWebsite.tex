Pengembangan Generator Konten \textit{Website} berbasis API bertujuan untuk merealisasikan sistem yang dirancang, menyediakan antarmuka API untuk pengelolaan templat, dan menghasilkan \textit{website} berdasarkan templat pilihan pengguna.

\begin{enumerate}[label*=\arabic*.,ref=\arabic*]
    \item Lingkungan Pengembangan
        \begin{itemize}
            \item \textbf{Bahasa Pemrograman:} JavaScript.
            \item \textbf{Runtime:} Node.js.
            \item \textbf{Framework Backend:} Express.js.
            \item \textbf{ORM:} \textit{Sequelize} untuk PostgreSQL.
            \item \textbf{Pustaka:}
                \begin{itemize}
                    \item \texttt{body-parser}: \textit{Middleware} untuk \textit{parsing body request.}
                    \item \texttt{dotenv}: Manajemen konfigurasi lingkungan.
                    \item \texttt{joi}: Validasi data.
                    \item \texttt{jsonwebtoken}: Otentikasi berbasis token.
                    \item \texttt{pg} dan \texttt{pg-hstore}: Koneksi PostgreSQL.
                \end{itemize}
        \end{itemize}
        
    \item Struktur Proyek\\
        Struktur proyek dirancang untuk mendukung modularitas dan kemudahan pengelolaan kode. Berikut adalah struktur direktori proyek:

        \begin{itemize}
            \item \textbf{config/}: Berisi file konfigurasi seperti koneksi basis data.
            \item \textbf{controllers/}: Berisi fungsi untuk menangani permintaan dan respons API.
            \item \textbf{routes/}: Mendefinisikan \textit{endpoint} API dan menghubungkannya dengan \textit{controller}.
            \item \textbf{models/}: Berisi definisi tabel basis data menggunakan Sequelize.
            \item \textbf{repositories/}: Mengabstraksi \textit{query} basis data untuk digunakan oleh \textit{service}.
            \item \textbf{services/}: Berisi logika bisnis, seperti pengelolaan templat atau proses \textit{generate website.}
            \item \textbf{middlewares/}: \textit{Middleware} untuk validasi data, otentikasi JWT, dll.
            \item \textbf{utils/}: Berisi fungsi pendukung yang digunakan di berbagai bagian sistem.
        \end{itemize}

        
    \item Implementasi Fitur
        \begin{enumerate}[label=\alph*.]
            \item \textbf{Pengelolaan Templat \textit{Website}}\\
            API ini memungkinkan pengguna untuk mengelola templat dengan menambahkan, membaca, memperbarui, dan menghapus templat. Contoh implementasi \textit{endpoint} untuk menambahkan templat baru:
            \begin{lstlisting}[language=Javascript,caption={Asset Static Route}]
const express = require("express");
const {
	AssetStaticController,
} = require("../controllers/AssetStaticController");
const { Authorization } = require("../middlewares/Authorization");
const router = express.Router();

router.use("/", Authorization.decryption());

router.get("/", AssetStaticController.getAll);
router.get("/:id", AssetStaticController.getOne);
router.post("/", AssetStaticController.post);
router.patch("/:id", AssetStaticController.patch);
router.delete("/:id", AssetStaticController.delete);

module.exports = router;
\end{lstlisting}

            \item \textbf{\textit{Generate Website}}\\
            API ini memungkinkan pengguna menghasilkan \textit{website} berdasarkan templat. Proses melibatkan penggabungan aset statis dan dinamis, serta penyimpanan hasil di direktori yang telah ditentukan. Contoh kode:
            \begin{lstlisting}[language=Javascript,caption={Generate Website Service}]
const { WebsiteRepository } = require("../repositories/WebsiteRepository");
const fs = require("fs");
const path = require("path");

// Fungsi untuk mendekode Base64
function decodeBase64(content) {
	return Buffer.from(content, "base64").toString("utf-8");
}

class GenerateService {
	static async generate(dataGenerate) {
		try {
			const findGenerate = await WebsiteRepository.readForGenerate(
				dataGenerate
			);

			const basePath = path.resolve(
				__dirname,
				"..",
				findGenerate.website.dir_path
			);

			await this.processAssets(findGenerate.assets, basePath);

			return { success: true };
		} catch (error) {
			throw error;
		}
	}

	static async processAssets(assets, basePath) {
		for (const asset of assets) {
			const assetPath = path.join(basePath, asset.name);

			if (asset.type === "folder") {
				// Buat folder jika belum ada
				if (!fs.existsSync(assetPath)) {
					fs.mkdirSync(assetPath, { recursive: true });
					console.log(`Folder created: ${assetPath}`);
				}
				// Rekursi untuk anak-anak dari folder ini
				if (asset.children && asset.children.length > 0) {
					await this.processAssets(asset.children, assetPath);
				}
			} else if (asset.type === "file") {
				// Decode content jika ada
				let content = asset.content ? decodeBase64(asset.content) : "";

				// Ganti placeholder dengan placeholder_values
				if (asset.placeholder_values) {
					for (const [key, value] of Object.entries(asset.placeholder_values)) {
						const placeholder = new RegExp(`{{${key}}}`, "g");
						content = content.replace(placeholder, value);
					}
				}

				// Handle backticks dalam konten
				if (content.includes("`")) {
					content = content.replace(/`/g, "`"); // Pastikan backticks tetap aman
				}

				// Tulis konten ke file
				fs.writeFileSync(assetPath, content, "utf-8");
				console.log(`File created: ${assetPath}`);
			} else {
				console.warn(`Unknown asset type: ${asset.type} for ${asset.name}`);
			}
		}
	}
}

module.exports = { GenerateService };
\end{lstlisting}

            \item \textbf{Validasi Data}\\
            Validasi dilakukan menggunakan pustaka \texttt{joi} untuk memastikan data sesuai dengan skema. Contoh implementasi validasi:
           \begin{lstlisting}[language=Javascript,caption={Validasi Masukan Data Aset Statis}]
createAssetStatic(data) {
   const schema = Joi.object({
      website_id: Joi.number().integer().min(1).required().messages({
         "number.base": "Website ID must be a number.",
         "number.integer": "Website ID must be an integer.",
         "number.min": "Website ID must be at least 1.",
         "any.required": "Website ID is required.",
      }),
      parent_id: Joi.number().integer().min(1).allow(null).optional().messages({
         "number.base": "Parent ID must be a number.",
         "number.integer": "Parent ID must be an integer.",
         "number.min": "Parent ID must be at least 1.",
      }),
      name: Joi.string().max(200).required().messages({
         "string.base": "Name must be a string.",
         "string.empty": "Name cannot be empty.",
         "string.max": "Name cannot exceed 200 characters.",
         "any.required": "Name is required.",
      }),
      type: Joi.string().valid("file", "folder").required().messages({
         "string.base": "Type must be a string.",
         "any.only": "Type must be either 'file' or 'folder'.",
         "any.required": "Type is required.",
      }),
      content: Joi.string().allow(null).optional().messages({
         "string.base": "Content must be a string.",
      }),
      placeholders: Joi.object().allow(null).optional().messages({
         "object.base": "Placeholders must be an object or null.",
      }),
   }).required();

   return schema.validate(data, { abortEarly: false });
}
\end{lstlisting}
        \end{enumerate}
        
\end{enumerate}