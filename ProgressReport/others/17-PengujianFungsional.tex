Pengujian fungsional bertujuan untuk memastikan bahwa setiap fitur dari Generator Konten Website berbasis API berfungsi sesuai dengan spesifikasi yang telah ditentukan. Pengujian dilakukan dengan menguji endpoint API menggunakan skenario pengujian yang mencakup berbagai kondisi masukan dan keluaran.

\begin{enumerate}[label*=\arabic*.,ref=\arabic*]
    \item Tujuan Pengujian
        \begin{itemize}
            \item Memastikan setiap endpoint API bekerja sesuai dengan spesifikasi.
            \item Memverifikasi validasi data masukan pada API.
            \item Memastikan proses \textit{generate website} menghasilkan output yang sesuai.
            \item Mengidentifikasi dan memperbaiki kesalahan dalam implementasi.
        \end{itemize}
    
    \item Lingkungan Pengujian\\
        Pengujian dilakukan pada lingkungan pengembangan dengan konfigurasi sebagai berikut:
        \begin{itemize}
            \item \textbf{Alat Pengujian:} Postman untuk Windows v11.22.1
            \item \textbf{Database:} PostgreSQL
            \item \textbf{Sistem Operasi:} Windows 11
            \item \textbf{Runtime:} Node.js v20.10.0
            \item \textbf{Dependensi:} express v4.21.2, express-session v1.18.1, joi v17.13.3, pg v8.13.1, pg-hstore v2.3.4, sequelize v6.37.5, body-parser v1.20.3, jsonwebtoken v9.0.2, axios v1.7.9, nodemon v3.1.7, dan dotenv v16.4.7
        \end{itemize}
    
    \item Skenario Pengujian dan Hasil\\
        Berikut adalah tabel skenario pengujian untuk setiap \textit{enpoint}:
        
\end{enumerate}